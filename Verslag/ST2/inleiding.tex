
Kleurdetectie is een belangrijke component in het herkennings- en detectiealgoritme van de {\it Screencaster}. Vooral het herkenningsalgoritme heeft een correcte kleurherkenning nodig. Er zijn vele manieren om kleuren te detecteren. Het verslag behandeld enerzijds de verschillen in kleurruimten, in het bijzonder HSL en RGB. Anderzijds behandeld het de verschillen in de omgeving en hoe deze effect hebben op het juist benoemen van de kleuren.

De focus van dit verslag ligt op hoe deze bevindingen het project kunnen verbeteren. De herkenning gebeurt nu op basis van kleurbereik in HSL. Dit is al besproken in vorige verslagen. In het besluit, meer bepaald in subsectie \ref{subsec:toepassingen}, gaat het verslag dieper in op de veranderingen die kunnen gebeuren om de schermdetectie te verbeteren.