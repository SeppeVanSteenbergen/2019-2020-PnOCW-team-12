Vervolgens kan ook gekeken worden of het toepassen van een filter op een onjuist gedetecteerde foto een al dan niet positief effect heeft.
Hiervoor kan onderscheid gemaakt worden tussen afbeeldingen die initieel juist en fout zijn gedetecteerd. 
In figuur \ref{fig:initieelfout} worden de proporties weergegeven die minder scoren na het toepassen van een filter op een initieel fout gedetecteerde afbeelding. Dit geeft dus het percentage foto's weer die nog steeds niet gedetecteerd worden na het filteren ervan.

\begin{figure}[h!]
  \includegraphics{img/initieelFout}
  \caption{Proportie slechter gedetecteerde foto's na foute initiële detectie.}
  \label{fig:initieelfout}
\end{figure}

Het is opmerkelijk hoe dicht alle waarden bij elkaar liggen. Ze leunen allen zeer dicht aan tegen de één. Dit wil dus zeggen dat de filters de resultaten niet verbeterd hebben, maar eigenlijk enkel slechter hebben gemaakt. Afbeeldingen die initieel slecht scoren zullen na de filtering geen significant beter resultaat opleveren. 

Daarnaast zijn er degene die initieel juist gedetecteerd zijn. Scoren deze beter indien er een filter op wordt toegepast?
Uit figuur \ref{fig:initieeljuist} wordt al snel duidelijk dat er hier een verband is met kerngrootte. Indien een kern van grootte zeven à negen wordt gebruikt zullen alle afbeeldingen een hogere score hebben en dit bij alle geteste filters. Bij een grootte van drie zal ongeveer acht procent van de foto's een hogere score hebben. Kernen van vijf en zeven zorgen voor een middelmaat van om en beide 40 en 63 procent respectievelijk. Tussen de filters onderling zijn bij gelijke kerngroottes geen significante verschillen op te merken. De waarden zijn zo goed als gelijk.  

\begin{figure}[h!]
  \includegraphics{img/initieeljuist}
  \caption{Proportie slechter gedetecteerde foto's na foute initiële detectie.}
  \label{fig:initieeljuist}
\end{figure}