 \subsection{Barcode}
Om de verschillende schermen te identificeren wordt gebruik gemaakt van een kleuren barcode. De barcode bestaat uit een herhalend patroon van 5 unieke kleuren telkens gevolgd door een witte lijn. Door het gebruik van deze witte lijn weet het algoritme waar het patroon eindigt en de volgende sequentie terug opnieuw begint. Het detecteren van deze 5 kleuren gebeurt aan de hand van opgestelde HSL ranges. Voor de identificatie van de slaves wordt er dus een unieke combinatie van deze 5 kleuren weergegeven. Deze vormt dan een unieke vijfcijferige code die gelinkt kan worden aan de bijhorende slave. Dit zorgt ervoor dat we in theorie een totaal van $5!$ ($=120$) verschillende schermen op één moment kunnen detecteren. Herhaling van het patroon heeft als resultaat dat bij overlap het scherm nog steeds geïdentificeerd kan worden. Het algoritme zal twee keer over alle pixels itereren. Hierdoor heeft het algorirmte een tijdscomplexiteit van
\[O(2mn)=O(mn)\]
met m en n de dimensies van het island waarin de barcode gelezen wordt. Het algoritme gaat een keer horizontaal en een keer verticaal om zo in alle mogelijke orientaties van het scherm de barcode te kunnen lezen. Vervolgens worden de HSL waarden van deze pixels bekeken om de overeenkomstige kleur van elke pixel te achterhalen. Wanneer een HSL waarde binnen de gewenste range valt, wordt het overeenkomstig cijfer opgeslaan in een lijst. Bij het bereiken van een witte lijn weet het algoritme dat het aan het einde van het patroon is. Wanneer dit het geval is wordt het inlezen van de volledige vijfcijferige code gecontroleerd op volledigheid. Zo niet wordt de lijst leeg gehaald en zoekt het algoritme verder. Het herhalend patroon is dus essentieel aan het correct inlezen van de barcode. Een groter aantal herhalingen stemt overeen met meer kansen op een mogelijke detectie, maar stemt ook overeen met een kleinere oppervlakte per herhaling. Deze kleinere oppervlakte is dan weer nadelig voor detectie omdat hiermee de kans stijgt dat een kleur niet gededecteerd wordt. Nadat het algoritme over alle pixels geweest is, wordt de ratio berekend tussen de code die het meeste keer gelezen is en de totale aantal codes die zijn gelezen. Deze ratio bepaalt dan of de code die door de horizontale iteratoe of die door de verticale iteratei het meest gelezen is, gebruikt wordt.

\subsection{Verdere verbeteringen}
Zoals al reeds vermeld bij het deel over kleurdetectie is het individueel dedecteren van zoveel verschillende kleuren geen goed idee. Dit is dan ook de reden dat andere opties reeds bekeken worden. Zoals ook reeds vermeld zou een eerste optie zijn om te kijken naar het contrast tussen de opeenvolgende kleuren in plaats van de kleuren apart te dedecteren. Een andere optie die bekeken wordt is om gewoon over te schakelen naar een zwart wit barcode waarbij eerder ook gebruik zal gemaakt worden van het contrast tussen zwart en wit dan de detectie ervan.
