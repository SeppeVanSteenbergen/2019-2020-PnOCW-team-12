De {\it basic slave screen detection} zoals beschreven in de opgave bij taak 3 is nagenoeg volledig geïmplementeerd en werkende. Waar er in het begin {\it openCV} werd gebruikt, zijn er nu eigen algoritmen die al het werk leveren. Verschillende schermen kunnen worden gedetecteerd al moet er voor het zoeken naar de hoeken van deze schermen wel nog een vernufter algoritme komen. De oriëntatie van de schermen wordt snel en correct gedetecteerd. Het algoritme om de oriëntatie te vinden, gaat wel uit van correct gevonden hoeken en kan dus nog verbeterd worden. De implementatie van de algoritmen is zo efficiënt en zo simpel mogelijk gehouden, er wordt vaak niet meer gedaan dan enkel over de aparte pixels gelopen.

De identificatie van de schermen gebeurd met een kleurenbarcode. Deze kan op een zeer klein scherm nog worden gedetecteerd. Aangezien de code herhaalt kan worden op het scherm is het ook mogelijk om bij gedeeltelijke bedenking een scherm te herkennen. Deze methode berust nu nog op de assumptie dat het scherm recht staat. Er is al een oriëntatiematrix opgesteld om het scherm naar behoren te draaien maar dit is nog niet geïmplementeerd.
\paragraph{}
De basis voor {\it basic slave screen detection} is gelegd. De komende weken zal de code nog wat opgeruimd moeten worden. Ongebruikte functies moeten verdwijnen en geheugenmanagement moet worden herbekeken. Ook voor scaling moet er nog gewacht worden. Deze kan nog niet herkent worden. Als later de communicatie mogelijk is tussen de {\it slaves} en de {\it master} zal de grootte van het scherm worden opgevraagd. Door deze te vergelijken met de grootte van het scherm, kan de scaling worden berekend. Er is nog veel werk aan de winkel maar de basis die er nu is, mag er toch al wel zijn.