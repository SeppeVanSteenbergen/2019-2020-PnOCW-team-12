\section{Transformatie voor image-show}
Alle schermen moeten een deel van een foto weergeven. Echter doordat de schermen drie--dimensionaal gedraaid zijn, moet de weer te geven foto getransformeerd worden. De afbeelding moet namelijk vanuit perspectief twee--dimensionaal zijn afgebeeld. Het gebruikte algoritme gaat vier bron-- en bestemmingshoeken gebruiken als invoer om tot een transformatie--matrix te komen. Deze matrix wordt gebruikt om later met {\it image3d} transform een snelle transformatie op een canvaselement mogelijk te maken.

\subsection{transformatiematrix}
De vier bron-- en bestemmingshoeken zijn respectievelijk de start-- en eindhoeken. Er zal per lijst van hoeken een matrix worden berekent nadat de coefficiënten berekend zijn geweest. $(x_i, y_i)$ zijn de hoekpunten, $i = 1..4$ voor de start--hoeken en $i = 5..8$ voor de eind--hoeken. \cite{redrawImageFrom3dPerspectiveTo2d}
$$ \begin{bmatrix}
x_1 & x_2 & x_3 \\ y_1 & y_2 & y_3 \\ 1 & 1 & 1
\end{bmatrix} \begin{bmatrix}
\lambda \\ \mu \\ \tau
\end{bmatrix} =
\begin{bmatrix}
x_4 \\ y_4 \\ 1
\end{bmatrix}
$$
$$ \begin{bmatrix}
\lambda \\ \mu \\ \tau
\end{bmatrix} =\begin{bmatrix}
x_1 & x_2 & x_3 \\ y_1 & y_2 & y_3 \\ 1 & 1 & 1
\end{bmatrix}^{-1}
\begin{bmatrix}
x_4 \\ y_4 \\ 1
\end{bmatrix}
$$
Daarna wordt de 3x3 matrix geschaald met de gevonden coefficiënten.
$$
\begin{bmatrix}
\lambda x_1 & \mu x_2 & \tau x_3 \\ \lambda y_1 & \mu y_2 & \tau y_3 \\ \lambda & \mu & \tau
\end{bmatrix}
$$
De twee matrices A en B, respecitivelijk met de bron- en bestemmingshoeken, worden gebruikt om de transformatiematrix te berekenen.
$$ C = AB^{-1}$$

$C$ is de transformatiematrix, deze wordt later gebruikt om de {\it matrix3d} op te stellen. \cite{projectiveTransformation}
$$
\begin{Bmatrix}
C_{0,0}	& 	C_{0,1}	&	0	&	C_{0,2} 	\\
C_{1,0}	&	C_{1,1} 	&	0	&	C_{1,2} 	\\
0		&	0		&	1	&	0		\\
C_{2,0}	&	C_{2,1}	&	0	&	C_{2,2}
\end{Bmatrix}
$$
