Vervolgens bekijkt het verslag of het toepassen van een filter op een onjuist gedetecteerde foto, een al dan niet positief effect heeft.
Hiervoor maakt het onderzoek een onderscheid tussen afbeeldingen die initieel juist of initieel fout zijn gedetecteerd. 
In figuur \ref{fig:initieelfout} worden de proporties weergegeven die minder scoren na het toepassen van een filter op een initieel fout gedetecteerde afbeelding. Dit geeft dus het percentage foto's weer die slechter scoort na het filteren ervan.

\begin{figure}[h!]
  \includegraphics[width=\linewidth]{img/initieelFout}
  \caption{Proportie slechter gedetecteerde foto's na foute initiële detectie.}
  \label{fig:initieelfout}
\end{figure}

Het is opmerkelijk hoe dicht alle waarden bij elkaar liggen. Ze leunen allen zeer dicht aan tegen de $100\%$. Dit wil zeggen dat de filters de resultaten enkel verslechteren bij een al fout gedetecteerde foto. Afbeeldingen die initieel slecht scoren zullen na de filtering geen beter, tot fouter resultaat opleveren. 

Daarnaast zijn er degene die initieel juist gedetecteerd zijn. Uit figuur \ref{fig:initieeljuist} wordt al snel duidelijk dat er hier een verband is met kerngrootte. Indien de kerngrootte zeven à negen bedraagt, zullen alle afbeeldingen een hogere score hebben en dit bij alle geteste filters. Bij een grootte van drie zal ongeveer acht procent van de foto's een hogere score hebben. Kernen van grootte vijf en zeven zorgen voor een middelmaat van respectievelijk om en bij de 40 en 63 procent. Tussen de filters onderling zijn bij gelijke kerngroottes geen significante verschillen op te merken, de waarden zijn zo goed als gelijk. Het slechtere resultaat bij een kerngrootte van 11 is te wijten aan een te grote verspreiding van de oneffenheden.

Deze resultaten wijzen erop dat gebruik van filters bij de herkenning in {\it ScreenCaster} geen meerwaarde zullen hebben. Bij een foute detectie verbeteren de filters niets en bij een juiste detectie is er nog kans op een slechter resultaat.

\begin{figure}[h!]
  \includegraphics[width=\linewidth]{img/initieeljuist}
  \caption{Proportie slechter gedetecteerde foto's na foute initiële detectie.}
  \label{fig:initieeljuist}
\end{figure}