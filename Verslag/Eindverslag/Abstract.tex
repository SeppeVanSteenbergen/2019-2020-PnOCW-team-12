Het laten samenwerken van verschillende schermen van computers, tablets, smartphones en dergelijke met toegang tot een moderne web-browser, om samen één volledig beeld te vormen. Dit alles gecontroleerd door één master toestel. De toestellen zijn geconnecteerd via een webapplicatie en worden geconfigureerd door één foto te nemen met het delegerende toestel. Wel of geen overlap van toestellen, sommige schermen dicht en sommige verder van de camera, recht of gekanteld, het weergegeven resultaat moet vlak tonen vanuit het standpunt van de gebruiker.\\[3mm]
Hiervoor moeten de verschillende toestellen eerst gedeteceerd en geïdentificeerd worden. Het volledige resultaat bereiken door het maken van één enkele foto via een standaard smartphone camera was vanaf de start prioritair. Dit zou het hinderlijke probleem van een inconsistente camerapositie door het nemen van verschillende foto’s meteen elimineren. Al snel werd gewerkt met voorbepaalde kleuren die de toestellen afbeelden. Hierop kan dan gefilterd worden met geassocieerde kleurenbreedtes. Op basis hiervan is het dan mogelijk de verschillende maskers af te leiden. Verdere algoritmes zoals het detecteren van de hoeken en het reconstrueren van afgedekte delen gebeuren aan de hand van deze maskers. De link tussen gevonden schermen en verbonden toestellen met de server, worden gelinkt door het lezen van een unieke kleuren-barcode die overeen komt met een gebruikers-id. Aan de hand van de bekomen toestelspecificaties bij  het gedetecteerde scherm wordt er een perspectief matrix opgesteld om een correct resultaat op elk scherm af te beelden. De getransformeerde afbeelding zal zowel continuïteit als visuele correctheid over de verschillende schermen heen garanderen.\\[3mm]
Tijdens het steeds verder uitbreiden van de grenzen van het algoritme is het duidelijk geworden dat kleurendetectie het grootste struikelblok was van de volledige methode. Deze aanpak viel al snel uit elkaar bij relatief kleine afwijkingen in de omgevingsbelichting en reflecties op de schermen. Enkel bij schermen en camera’s die gebruikt werden voor het afstellen van de kleurdetectie werd een consistent resultaat verkregen. Een eenvoudige oplossing is om de barcode voor identificatie te beperken tot grijswaarden. Zo kan een grotere rijkweidte om de randen en diagonalen van het scherm te detecteren toegelaten worden. De ontworpen algoritmes voor hoekdetectie en reconstructie werken consistent bij een correct gevormd masker.
