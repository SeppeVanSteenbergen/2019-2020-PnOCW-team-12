
Kleurdetectie is een belangrijke component in het identificatie- en detectiealgoritme van de {\it Screencaster}. Vooral het identificatiealgoritme heeft nood aan een correcte interpretatie van kleur. Om een concretere kennis te verkrijgen binnen dit gebied van het project is een collectie aan beeldmateriaal verzameld. Aan de hand van deze gegevens zullen beslissingen genomen kunnen worden zoals welke kleuren best gebruikt worden. Er zijn verscheidene manieren om kleuren te detecteren. Het verslag behandelt enerzijds de verschillen in kleurruimten. Hierbij wordt vooral dieper ingegaan op HSL en RGB. Anderzijds behandelt het de verschillen in omgeving, belichting en helderheid waarbij gekeken wordt hoe deze effect hebben op het al dan niet juist identificeren van de kleuren.

De focus van dit verslag ligt op hoe deze bevindingen het project kunnen verbeteren. De herkenning gebeurt momenteel op basis van een kleurbereik in HSL. Dit is alreeds uitvoerig besproken in vorige verslagen. In het besluit, meer bepaald subsectie \ref{subsec:toepassingen}, wordt dieper ingegaan op de veranderingen die kunnen gebeuren om de schermdetectie te verbeteren.