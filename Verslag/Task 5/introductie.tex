Wanneer schermen herkent en geïdentificeerd zijn, moeten ze ook nog aan elkaar gelinkt worden. Hiervoor zal eerst elk scherm zijn eigen positie toegewezen krijgen. Daarna kan aan de hand van deze positie zijn buren bepaald worden. Dit kan dan gebruikt worden om bijvoorbeeld een slang van scherm naar scherm te laten bewegen. Delaunay triangulatie is hiervoor een goede manier om de juiste buren te vinden. Dit verslag behandeld de keuzes die gemaakt zijn alsook een uitleg bij het gebruikte algoritme en zijn tijdscomplexiteit. De mogelijke beperkingen worden met deze kennis geduid.