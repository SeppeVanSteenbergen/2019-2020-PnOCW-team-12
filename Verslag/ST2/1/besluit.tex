 
Uit de bevindingen is gebleken dat de interpretatie van kleur sterk afhankelijk is van de manier waarop kleur wordt voorgesteld en de omgevingsfactoren die hierbij aanwezig zijn. Enerzijds speelt de kleurruimte waarin een kleur wordt voorgesteld een grote rol bij de detectie. Van de verschillende kleurruimten die bekeken zijn, is er niet één die eenduidig beter is dan de andere. Beiden hebben hun voordelen, zo is HSL beter voor de detectie van alle kleuren dan RGB. Maar de detectieratio van zwart en wit is dan wel weer aannoemelijk beter in RGB dan in HSL.  Anderzijds spelen omgevingsfactoren ook een belangrijke rol bij de detectie van de kleuren. Hiervoor werd gekeken naar zowel de omgeving, de lichtinval en de helderheid van het scherm. De belangrijkste bevindingen hierbij hebben te maken met de lichtinval en de omgeving. Door lichtinval komt er bij artificieel licht extra rood in de kleuren door de rode schijn die het licht achterlaat. Hoe groter de omgeving, hoe meer de kleuren fout gedetecteerd gaan worden. Dit is te wijten aan het feit dat de camera hierbij op de omgeving gaat focussen in plaats van op het scherm. Tot slot is ook gebleken dat helderheid geen groot effect heeft op de detectie.