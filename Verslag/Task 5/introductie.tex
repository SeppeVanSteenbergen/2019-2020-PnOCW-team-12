Wanneer schermen herkent en geïdentificeerd zijn, moeten ze ook nog aan elkaar gelinkt worden. Hiervoor zal eerst elk scherm zijn eigen positie toegewezen krijgen. Daarna kunnen zijn buren bepaald worden met behulp van deze positie. Dit geeft ons de nodige informatie om bijvoorbeeld een slang van scherm naar scherm te laten bewegen. Delaunay triangulatie is een goede manier om de correcte buren te vinden. Dit verslag behandeld de keuzes die gemaakt zijn alsook een uitleg bij het gebruikte algoritme en zijn tijdscomplexiteit. De mogelijke beperkingen worden met deze kennis geduid.