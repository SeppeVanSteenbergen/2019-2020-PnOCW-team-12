\section{Inleiding}
De wereld wordt alsmaar digitaler. Mensen ontmoeten elkaar digitaal, spreken elkaar digitaal, zien elkaar digitaal. Ook het buitenspelen lijkt verleden tijd, samen een spelletje spelen online is ook geen rariteit meer. De digitale wereld past zich in sneltempo aan en zo moeten de toepassingen meegroeien. Wanneer mensen elkaar online willen zien of online een spel tegen elkaar willen spelen, willen ze dit het liefst zo echt mogelijk meemaken. Dit door een zo groot mogelijk scherm. Niet iedereen beschikt echter over de capiciteiten om een 100'' scherm aan te kopen. Een mogelijkheid die in dit verslag wordt besproken is het doen samenwerken van schermen. Eender welk scherm met een internetverbinding zou zo aan elkaar gekoppeld kunnen worden. Op de geconnecteerde schermen is het dan mogelijk de achtergronden in te stellen op een gekozen kleur, pijlen te tekenen, synchroon af te tellen vanaf een gekozen getal, een afbeelding of video te vertonen en zelf een ''kat en muis'' animatie weer te geven.
\bigskip
Het verslag gaat eerst dieper in op het framework van de online applicatie. Hoe communiceren master en clients? Welke  protocollen worden gebruikt? Beveiliging en dergelijke worden hier beantwoordt \ref{sec:framework}. Vervolgens komt synchronisatie van de schermen aan bod in sectie \ref{sec:synchronisatie}, dit is van groot belang bij het synchroon laten aftellen. Hierna wordt de detectie van de schermen besproken in sectie \ref{sec:detectie}, de schermen zullen gedetecteerd worden door één enkele foto. Wanneer schermen overlappen zullen er hoeken niet zichtbaar zijn. Daarom zal het scherm gereconstrueerd moeten worden. Hiervoor wordt verondersteld dat per scherm het middelpunt en twee aanliggende hoekpunten zichtbaar zijn. Het verloop hiervan wordt besproken in sectie \ref{sec:reconstructie}. Hierna volgt nog de identificatie in sectie \ref{sec:identificatie}. Waar en hoe is het scherm gesitueerd en georiënteerd op de ingezonden afbeelding? Alsook de transformatie van de af te beelden objecten komen aan bod. Er wordt geëindigd met triangulatie en vervolgens een kleine animatie die al op de schermen vertoond kan worden. Dit om te tonen dat de schermen juist met elkaar verbonden zijn en er een vloeiende overgang mogelijk is. 