\section{Testen}
Voor de correctheid van het Bowyer-Watson algoritme zijn bepaalde subalgorimen heel belangrijk. Zo is er een subalgoritme om te controleren of een punt wel degelijk binnen een cirkel ligt. Deze werd getest met een paar cases waar het punt in de cirkel, op de cirkel en uit de cirkel ligt.
Bij het subalgoritme waar twee driehoeken gevormd worden die alle punten omvatten wordt er gekeken of elk punt in één van beide ligt. Hierbij wordt gebruik maakt van meetkundige stellingen. Een punt ligt binnen een driehoek wanneer het voor elke zijde van deze driehoek aan dezelfde kant ligt, bij het doorlopen van deze zijden in éénzelfde richting. Dit wordt gecontroleerd aan de hand van het kruisproduct tussen de vector van de zijde en de vector van het punt. \cite{toepMeetkunde}