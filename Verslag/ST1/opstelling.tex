In de opstelling om de testen uit te voeren is er een server gebruikt waar alle clients zich op aansluiten aan de hand van een socket protocol. Met een knop op de pagina worden alle klokken gesynchroniseerd met de server door het tijdsverschil en de vertraging (ping) tussen de laatste twee te berekenen.
Het tijdsverschil wordt namelijk 10 keer herberekend om daaruit mogelijke verbindingsfluctuaties uit te kunnen filteren (meer hierover in sectie \ref{sec:testen}).

\begin{figure}[h]
\centering
\includegraphics[scale=0.4]{img/server-client-sync.jpg}
\caption{NTP Protocol} \label{serv-client}
\end{figure}

Voor de synchronisatie is het NTP protocol geïmplementeerd. \cite{NTP:1}
Het schema over het NTP protocol is te vinden in figuur \ref{serv-client}. TSx en TCx zijn de tijden (in milliseconden) gemeten door de server en respectievelijk de client.
Het tijdsverschil wordt gemeten door.
\newline
\[ timeDiff = \frac{(TC1 - TS1) + (TS2 - TC2)}{2}  \]

De vertraging voor de server om een bericht te verzenden en terug te ontvangen is:

\[ ping  = (TS2 - TS2) - (TC2 - TC1) \]


Hiermee is het makkelijk de servertijd te berekenen door de vergelijking:

\[ timeServer = timeClient - timeDiff \]

Om vervolgens de tijdsverschil tussen de server- en de wereldtijd te berekenen is er gebruikt gemaakt van een externe API, worldtimeapi.org. Het tijdsverschil is als volgt berekend:

\[ serverWorldOffset = (TW - TS4) + \frac{TS4 - TS3}{2} \]

Uiteindelijk is de volgende informatie opgeslagen:

\begin{itemize}
  \item lijst van ping tussen client en server
  \item standaardafwijking van de ping
  \item tijdsverschil tussen client en server tijd
  \item tijdsverschil server en wereldtijd
  \item tijdsverschil client en wereldtijd
\end{itemize}




