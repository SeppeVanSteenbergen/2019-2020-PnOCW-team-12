\documentclass[a4paper,11pt]{article}

\usepackage[margin=3cm]{geometry}

\usepackage{graphicx}
\usepackage{subcaption}
\usepackage[colorlinks,allcolors=violet]{hyperref}
\usepackage{url}
\usepackage{lmodern}
\usepackage[dutch]{babel}

% https://tex.stackexchange.com/questions/94032/fancy-tables-in-latex
\usepackage[table]{xcolor}
\usepackage{booktabs}

\usepackage[utf8]{inputenc}

% https://tex.stackexchange.com/questions/664/why-should-i-use-usepackaget1fontenc
\usepackage[T1]{fontenc}
\usepackage{microtype} % good font tricks

\newcommand{\note}[1]{{\colorbox{yellow!40!white}{#1}}}
\newcommand{\exampletext}[1]{{\color{blue!60!black}#1}}

\begin{document}

\noindent
\colorbox[HTML]{52BDEC}{\bfseries\parbox{\textwidth}{\centering\large
  --- Verslag P\&O CW 2019--2020 Taak 5 ---
}}
\\[-1mm]
\colorbox[HTML]{00407A}{\bfseries\color{white}\parbox{\textwidth}{
  Department Computerwetenschappen -- KU Leuven
  \hfill
  \today
}}
\\

\smallskip

\noindent

\begin{tabular}{*4l}
\toprule
\multicolumn{2}{l}{\large\textbf{Team 12}} \\
\midrule
Frédéric Blondeel & h \\
Martijn Debeuf & h \\
Toon Sauvillers & h \\ % fill in the time spend on this task per team member who worked on it
Dirk Vanbeveren & h \\
Bert Van den Bosch & h \\
Seppe Van Steenbergen & h \\


\bottomrule
\hline
\end{tabular}\\

\noindent
{\color[HTML]{52BDEC} \rule{\linewidth}{1mm} }

\section{Introductie}\label{sec:introductie}


\section{Algoritmen}
\subsection{Algemene uitleg}
De manier van aanpak werd dus veranderd. In plaats van een border langsheen de rand van het scherm te creëren is geopteerd geweest om de diagonalen te verbinden (een groene en een blauwe) en op het middelpunt wordt een lichtblauwe bol weergegeven.



\section{Besluit}\label{sec:besluit}

\newpage

\bibliographystyle{unsrt}
\bibliography{bibliography}



\end{document}