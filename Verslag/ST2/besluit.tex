 
Uit de bevindingen is gebleken dat de interpretatie van kleur sterk afhankelijk is van de manier waarop kleur wordt voorgesteld en de omgevingsfactoren die hierbij aanwezig zijn. Enderzijds speelt de kleurruimte waarin een kleur wordt voorgesteld dus een grote rol voor de detectie hiervan. Hieruit kan geconcludeerd worden dat er niet een kleuruimte is dat beter is dan de andere maat da beiden hun voordelen hebben. Zo is HSL beter voor het dedecteren van alle kleuren dan RGB. Maar is de detectieratio van zwart en wit dan wel weer aannoemelijk beter in RGB dan in HSL. Anderzijds spelen omgevingsfactoren ook een belangrijke rol bij het dedecteren van kleuren. Hierbij werden zowel de omgeving, lichtinval en helderheid van het scherm bekeken. De belangrijkste bevindingen hierbij zijn vooral dat bij lichtinval van artificieel licht er extra rood in de kleuren komen door de rode schijn die het licht achterlaat en dat hoe groter de omgeving wordt, hoe meer de kleuren fout gededecteerd gaan worden. Dit is te wijten aan het feit dat de camera hierbij op de omgeving gaat focussen in plaats van op het scherm. Tot slot is ook gebleken dat helderheid geen groot effect heeft op de detectie.