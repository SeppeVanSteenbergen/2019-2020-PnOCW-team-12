\section{Testen}
Voor de correctheid van het Bowyer-Watson algoritme zijn bepaalde subalgorimen heel belangrijk. Zo is er een subalgoritme om te controleren dat een punt wel degelijk in een cirkel ligt. Deze wordt getest met een paar cases waar het punt in de cirkel, op de cirkel en uit de cirkel ligt.
Voor het subalgoritme waar er twee driehoeken gevormd worden die alle punten omvatten wordt er gekeken dat alle punten in één van de twee ligt. Hierbij wordt meetkunde gebruikt. Een punt ligt in een driehoek wanneer het aan dezelfde kant ligt van alle zijden van de driehoek, met alle zijden in éénzelfde richting doorlopen. Dit wordt gecontroleerd met behulp van het kruisproduct tussen de vector van de zijde en de vector van het punt. \cite{toepMeetkunde}