Voor het detectiealgoritme kan het best gebruik gemaakt worden van de HSL kleurruimte zoals nu al het geval is. Maar voor het identificatiealgoritme waarbij wit en zwart gebruikt wordt, zou beter overgeschakeld worden naar een algoritme dat gebruik maakt van de RGB kleurruimte.

De keuze voor de drie kleuren die weergegeven worden op het detectiescherm valt op rood, groen en blauw. Deze drie kleuren hebben een grote detectieratio en de omgevingsfactoren hebben het minste invloed op deze kleuren.

Aan de hand van de bevindingen rondom omgeving moeten foto's zo dicht mogelijk getrokken worden bij de opstelling zodat de omgeving minimaal is. Natuurlijk moet wel heel de opstelling in de foto passen.

