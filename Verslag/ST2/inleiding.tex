Het vorige verslag behandelde de kleurdetectie, gefocust op de te herkennen kleur en de invloed van omgevingsfactoren. Bij het herkennen van kleuren spelen oneffenheden in de foto's ook mee, dit verslag zal deze oneffenheden bespreken. Een foto is onderhevig aan verschillende factoren, zo zal een artificiële intelligentie die vaak aanwezig is in een camera de foto al bewerken. Lichtinval en een ander scherm of zelfs vuil op dit scherm maakt dat de kleuren niet egaal worden getoond.

Dit verslag behandeld verscheidene filters om alle ruwheid uit een foto weg te halen. In het eerste deel komen de verschillende filters aan bod, wat doen ze, hoe zijn deze geïmplementeerd... Nadat er meer uitleg gegeven is over de toegepaste methode, wordt er meer uitleg geveven over de bevindingen van dit onderzoek. Zowel de tijd waarin de filters werken, de score van de filters (zie sectie \ref{sec:methode} over het scoresysteem) als de efficiëntie van de filters komen aan bod.