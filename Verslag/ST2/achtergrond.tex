Er bestaat natuurlijk een breed gamma aan fotofilters. De benodigde code voor dit onderzoek is geschreven geweest in Python. Daarom is gekozen om vier filters uit de bibliotheek van OpenCV te gebruiken ( zie sectie \ref{sec:methode}).  Een eerste toegepaste filter is de \textit{Gaussian Blur}. Gevolgd door \textit{Median Blur} en \textit{Mean Blur}. Ten slotte is er ook een minder gekende ontruismethode gebruikt namelijk de \textit{Fast Non-Local Mean Denoising}.

\subsection{Gaussian Blur}
Deze methode filtert de hoge frequenties uit de foto en heeft dus het effect van een laagdoorlaatfilter. Er wordt een kernel grootte vooropgesteld. Deze grootte is de lengte van de matrix van pixels die bekeken wordt. Bijvoorbeeld indien de grootte 3 is zal de foto onderverdeeld worden in vierkanten van 3x3 pixels. Op elk van deze vierkanten wordt de methode dan toegepast. Binnen 
\subsection{Median Blur}

\subsection{Mean Blur}

\subsection{Fast Non-Local Mean Denoising}
