\section{Algemeen verloop detectie algoritme}
Er gaat heel wat vooraf aan het tonen van een afbeelding op alle cliënts. In bijlage \ref{bijlageA1} is een algemeen overzicht te vinden.
Het begint allemaal bij de afbeelding met alle schermen op, deze zal geanaliseerd worden. Om beter met kleuren te werken, wordt de afbeelding van {\it RGBA} naar het {\it HSLA} spectrum worden gebracht. Vervolgens wordt er gefilterd op de randkleuren van het detectiescherm. Hieruit worden de zogenaamde {\it islands} gehaald die gefilterd worden tot er enkel mogelijke schermen overblijven.
\paragraph{Hoekdetectie}
Vervolgens zoekt het algoritme naar hoeken, zie bijlage \ref{bijlageA2}. Er wordt gekeken of het scherm gekanteld of recht staat en met deze info het bijpassende hoekdetectie algoritme toegepast. Hierna worden de hoeken gefilterd en zijn de hoeken gedetecteerd. Als er niet genoeg hoeken gevonden zijn, is het geen geldig {\it island}.

\paragraph{Hoekreconstructie}
In bijlage \ref{bijlageA3} en deel \ref{sec:reconstructie}, staat het hoekreconstructie gedeelte gedetailleerder beschreven. Dit wordt uitgevoerd wanneer er niet genoeg hoeken zijn herkend voor een scherm, maar wel genoeg om te reconstrueren. Het gaat de missende hoeken reconstrueren door de lijnen vanuit de reeds gevonden hoeken te volgen.
\paragraph{}
Met alle hoeken op zak is er een scherm gevonden en wordt het geïdentificeerd met behulp van een barcode. Het scherm wordt toegewezen aan een cliënt. Het zoeken gaat verder tot alle cliënts gevonden zijn of alle pixels uit de afbeelding zijn overlopen.

