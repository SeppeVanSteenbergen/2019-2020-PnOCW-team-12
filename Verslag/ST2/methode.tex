Het onderzoek is volledig geautomatiseerd geweest aan de hand van Python. Binnen Python is gebruikt gemaakt van de bibliotheek OpenCv om de filters toe te passen op de afbeeldingen. De testen zijn uitgevoerd geweest op een gegevensbank met 670 afbeeldingen van schermen. Net zoals in het vorige onderzoek is gebruik gemaakt van verschillende omgevingsfactoren. Echter worden deze binnen dit onderzoek achterwege gelaten. 

Om de kwalitatieve data te kwantificeren is gewerkt met een scoresysteem. Een foto wordt voordat deze geanalyseerd wordt, onderverdeeld in een aantal blokken. Daarna worden binnen elk blok tien procent van de pixels willekeurig gekozen. Het onderverdelen in blokken zorgt ervoor dat de pixels mooi verspreid liggen binnen de afbeelding. Daarna worden de verschillende methodes op de foto toegepast. Om uiteindelijk een score aan de afbeeldingen toe te wijzen worden de eerder willekeurig bepaalde pixels bekeken. De score geeft weer welk percentage van deze pixels juist gedetecteerd zijn geweest, zonder en na het uitvoeren van elke filter.