\documentclass[a4paper,11pt]{article}

\usepackage[margin=3cm]{geometry}


\usepackage{graphicx}
\usepackage{subcaption}
\usepackage[colorlinks,allcolors=violet]{hyperref}
\usepackage{url}
\usepackage[dutch]{babel}



% https://tex.stackexchange.com/questions/94032/fancy-tables-in-latex
\usepackage[table]{xcolor}
\usepackage{booktabs}

\usepackage[utf8]{inputenc}

% https://tex.stackexchange.com/questions/664/why-should-i-use-usepackaget1fontenc


\newcommand{\note}[1]{{\colorbox{yellow!40!white}{#1}}}
\newcommand{\exampletext}[1]{{\color{blue!60!black}#1}}

\begin{document}

\noindent
\colorbox[HTML]{52BDEC}{\bfseries\parbox{\textwidth}{\centering\large
  --- Report P\&O CW 2019--2020 Task ST2.2 ---
}}
\\[-1mm]
\colorbox[HTML]{00407A}{\bfseries\color{white}\parbox{\textwidth}{
  Department of Computer Science -- KU Leuven
  \hfill
  \today
}}
\\

\smallskip

\noindent
\begin{tabular}{*4l}
\toprule
\multicolumn{2}{l}{\large\textbf{Team 12}} \\
\midrule
Martijn Debeuf &  22h\\ % fill in the time spend on this task per team member who worked on it
Toon Sauvillers &  22h\\
Seppe Van Steenbergen & 11h\\
\bottomrule
\hline
\end{tabular}\\

\noindent
{\color[HTML]{52BDEC} \rule{\linewidth}{1mm} }

\tableofcontents
\newpage
\section{Inleiding}\label{sec:inleiding}
	
Kleurdetectie is een belangrijke component in het herkennings- en detectiealgoritme van de {\it Screencaster}. Vooral het herkenningsalgoritme heeft een correcte kleurherkenning nodig. Er zijn vele manieren om kleuren te detecteren. Het verslag behandeld enerzijds de verschillen in kleurruimten, in het bijzonder HSL en RGB. Anderzijds behandeld het de verschillen in de omgeving en hoe deze effect hebben op het juist benoemen van de kleuren.

De focus van dit verslag ligt op hoe deze bevindingen het project kunnen verbeteren. De herkenning gebeurt nu op basis van kleurbereik in HSL. Dit is al besproken in vorige verslagen. In het besluit, meer bepaald in subsectie \ref{subsec:toepassingen}, gaat het verslag dieper in op de veranderingen die kunnen gebeuren om de schermdetectie te verbeteren.

\section{Achtergrond}\label{sec:achtergrond}
	Er bestaat natuurlijk een breed gamma aan fotofilters. De benodigde code voor dit onderzoek is geschreven geweest in Python. Daarom is gekozen om vier filters uit de bibliotheek van OpenCV te gebruiken ( zie sectie \ref{sec:methode}).  Een eerste toegepaste filter is de \textit{Gaussian Blur}. Gevolgd door \textit{Median Blur} en \textit{Mean Blur}. Bij deze methodes wordt gekeken naar de pixels rond een centrale pixel om zijn waarde te bepalen. Ten slotte is er ook een minder gekende ontruismethode gebruikt namelijk de \textit{Fast Non-Local Mean Denoising}. Deze werkt anders dan de eerste drie.

\subsection{Gaussian Blur}
Deze methode filtert de hoge frequenties uit de foto en heeft dus het effect van een laagdoorlaatfilter. Er wordt een kernel grootte vooropgesteld. Deze grootte is de lengte van de matrix van pixels die bekeken wordt. Bijvoorbeeld indien de grootte 3 is zal de foto onderverdeeld worden in vierkanten van 3x3 pixels. Op elk van deze vierkanten wordt de methode dan toegepast. NOG UITLEGGEN HOE GAUSSIAN PRECIES WERKT!!!!!!!!!!!!!!!!!!!!

\subsection{Median Blur}
Binnen de kernel van pixels worden de waarden van de pixels gesorteerd van klein naar groot, daarna wordt de middelste waarde (mediaan) geselecteerd. De kleur van de centrale pixel van de kernel wordt dan op deze waarde geplaatst.

\subsection{Mean Blur}
Bepaald binnen de kernel de gemiddelde waarde van de pixels. Daarna wordt de kleur van de centrale pixel op deze waarde geplaatst.

\subsection{Fast Non-Local Mean Denoising}
Aan  de hand van de kernel worden binnen de afbeelding gelijkende kernels gezocht. Gelijkende kernels worden bij elkaar geplaatst. Van elk van deze groepen wordt dan het gemiddelde bepaald. Daarna wordt elke kernel binnen die groep op de gemiddelde waarde geplaatst.  %Alle gebruikte filters bespreken

\section{Methode}\label{sec:methode}
	Om de kwalitatieve data te kwantificeren is gewerkt met een scoresysteem. Een foto wordt voordat deze geanalyseerd wordt, onderverdeeld in een aantal blokken. Daarna worden binnen elk blok tien procent van de pixels willekeurig gekozen. Het onderverdelen in blokken zorgt ervoor dat de pixels mooi verspreid liggen binnen de afbeelding. Daarna worden de verschillende methodes op de foto toegepast. Om uiteindelijk een score aan de afbeeldingen toe te wijzen worden de eerder willekeurig bepaalde pixels bekeken. De score geeft weer welk percentage van deze pixels juist gedetecteerd zijn geweest, zonder en na het uitvoeren van elke filter.

\section{Bevindingen}\label{sec:bevindingen}
	Nu kan gekeken worden of het al dan niet interessant is om al dan niet de afbeeldingen te filteren. 
De eerste belangrijke indicator was tijdsconsumptie. Daarnaast kunnen de filters onderling vergeleken worden. Vervolgens kan gekeken worden of de kernel grootte een invloed heeft op de resultaten. Tenslotte wordt nagegaan  of filteren effectief het detecteren van de kleur positief beïnvloed. 

	\subsection{Tijdsconsumptie}\label{subsec:tijd}
		Wanneer gekeken wordt naar figuur \ref{fig:tijd} valt meteen op dat \textit{Fast Non-Local Mean Denoising} voor om het even welke kern grootte niet echt efficiënt is. Dit kon alreeds verwacht worden door de manier waarop deze methode te werk gaat. De waarde blijft nagenoeg constant geacht de kern grootte maar is trager met een orde van grootte 3 ten opzichte van de andere filters. Om deze reden zal deze filter al zeker niet opgenomen worden in het project. 

\begin{figure}[h!]
    \centering
    \includegraphics[scale=0.5]{img/tijdsconsumptie}
    \caption{Gemiddelde tijd nodig om filter op afbeelding toe te passen.}
    \label{fig:tijd}
\end{figure}

Voor de andere filters kunnen hun individuele tijdsgrafieken \ref{fig:tijdsgrafieken} van dichterbij bekeken worden. 
Bij de \textit{Gaussian filter} is een duidelijk lineair verband tussen kerngrootte en tijdsconsumptie zichtbaar. Dit wordt bevestigd door de determinatiecoëfficiënt van 97,85 procent van de lineaire trendlijn. 


Verder zijn er nog twee opvallende zaken. Enerzijds zorgt een grotere kern ervoor dat de \textit{Median filter} veel meer tijd nodig heeft. De reden hiervoor is dat hoe groter de kern, hoe meer elementen pixelwaarden gesorteerd moeten worden van van klein naar groot.

Bij de \textit{Mean filter} houden de waarden er nagenoeg constante waarde. Rond kern grootte 6 gebeurt een sprong in uitvoeringstijd. 
WHY??????????????????????????

 \begin{figure}[h!]
  \centering
  \begin{subfigure}{0.4\linewidth}
    \includegraphics[width=\linewidth]{img/gaussiantijd}
  \end{subfigure}
  \begin{subfigure}{0.4\linewidth}
    \includegraphics[width=\linewidth]{img/mediantijd}
  \end{subfigure}
  \begin{subfigure}{0.4\linewidth}
    \includegraphics[width=\linewidth]{img/meantijd}
  \end{subfigure}
  \caption{Tijdsgrafieken voor de verschillende filters}
  \label{fig:tijdsgrafieken}
\end{figure}

Wanneer naar de tijdscomponent gekeken wordt geven \textit{Gaussian en Median filter} de beste resultaten. 
Op basis van tijd wordt de laatste methode al afgeschreven. 




	\subsection{Juiste herkenning}\label{subsec:juisteherkenning}
		Vervolgens bekijkt het verslag of het toepassen van een filter op een onjuist gedetecteerde foto een al dan niet positief effect heeft.
Hiervoor maakt het onderzoek een onderscheid tussen afbeeldingen die initieel juist of initieel fout zijn gedetecteerd. 
In figuur \ref{fig:initieelfout} worden de proporties weergegeven die minder scoren na het toepassen van een filter op een initieel fout gedetecteerde afbeelding. Dit geeft dus het percentage foto's weer die slechter scoort na het filteren ervan.

\begin{figure}[h!]
  \includegraphics[width=\linewidth]{img/initieelFout}
  \caption{Proportie slechter gedetecteerde foto's na foute initiële detectie.}
  \label{fig:initieelfout}
\end{figure}

Het is opmerkelijk hoe dicht alle waarden bij elkaar liggen. Ze leunen allen zeer dicht aan tegen de $100\%$. Dit wil zeggen dat de filters de resultaten enkel verslechteren bij een al fout gedetecteerde foto. Afbeeldingen die initieel slecht scoren zullen na de filtering geen beter, tot fouter resultaat opleveren. 

Daarnaast zijn er degene die initieel juist gedetecteerd zijn. Uit figuur \ref{fig:initieeljuist} wordt al snel duidelijk dat er hier een verband is met kerngrootte. Indien de kerngrootte zeven à negen bedraagt, zullen alle afbeeldingen een hogere score hebben en dit bij alle geteste filters. Bij een grootte van drie zal ongeveer acht procent van de foto's een hogere score hebben. Kernen van vijf en zeven zorgen voor een middelmaat van om en beide 40 en 63 procent respectievelijk. Tussen de filters onderling zijn bij gelijke kerngroottes geen significante verschillen op te merken, de waarden zijn zo goed als gelijk. Het slechtere resultaat bij een kernelgrootte van 11 is te wijten aan een te grote verspreiding van de oneffenheden.

Deze resultaten wijzen erop dat gebruik van filters bij de herkenning in {\it ScreenCaster} geen meerwaarde zullen hebben. Bij een foute detectie verbeteren de filters niets en bij een juiste detectie is er nog kans op een slechter resultaat.

\begin{figure}[h!]
  \includegraphics[width=\linewidth]{img/initieeljuist}
  \caption{Proportie slechter gedetecteerde foto's na foute initiële detectie.}
  \label{fig:initieeljuist}
\end{figure}
	\subsection{Gefilterd en ongefilterd} \label{subsec:gefilterd en ongefilterd}
		Om het uiteindelijke oordeel te vellen of filteren al dan niet een verbeterende factor kan zijn binnen het project, doet het verslag nog een laatste vergelijking. Namelijk kijken naar het verschil in juist gedetecteerde afbeeldingen per gebruikte methode alsook zonder het gebruik van een filter.

\begin{figure}[h!]
  \includegraphics[width=\linewidth]{img/filternofilter}
  \caption{Aantal juist gedetecteerde foto's van de 670, gegeven de gebruikte filter.}
  \label{fig:filternofilter}
\end{figure}

Ook hier zijn de waarden nagenoeg gelijk, van de 670 afbeelding worden er bij elke toegepaste filter en gebruikte kerngrootte gemiddeld 495 correct gedetecteerd. Dit wijkt slechts af met maximaal één. Hieruit volgt alweer dat een filter toepassen nagenoeg geen effect heeft op het detecteren van de kleuren.

\section{Besluit}\label{sec:besluit}
	De basis functionaliteiten zijn al op hun plek en werken al aanzienlijk goed samen. Een belangrijk onderdeel voor een accurate projectie van de foto is de detectie van het scherm en het reconstrueren van overlappende en verborgen delen van een scherm. De huidige kleurenfilters voor de detectie gaan nog verder uitgewerkt worden aan de hand van een verbeterde threshold formule die meerde kanalen van het HSL--spectrum gebruikt. Ook de reconstructie methode gaat nog uitgebreid en verfijnd worden om niet enkel meer variërende situaties op te lossen, maar ook een nauwkeuriger resultaat te leveren.
	
	\subsection{Toepassingen op het project}\label{subsec:toepassingen}
		Door het opvallende resultaat dat bij kleine oneffenheden een filter enkel nadelig is, zal de gebruikte filtermethode in de {\it ScreenCaster} herbekeken worden. Bij deze foto's is er echter een grotere vorm van oneffenheid, dit vereist specifieker onderzoek. Indien bij de applicatie een filter geen meerwaarde blijkt te hebben, zal deze geschrapt worden met een sneller reagerende applicatie tot gevolg.

\newpage

\bibliographystyle{unsrt}
\bibliography{Bibliography}

\end{document}