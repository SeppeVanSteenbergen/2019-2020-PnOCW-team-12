Het vinden van oriëntation maakt gebruik van de hoeken. Na het zoeken van de hoeken wordt de linkse hoek en de bovenste hoek gebruikt als vector. De hoek tussen deze vector en de x-as wordt berekend. Om vervolgens de precieze oriëntatie te weten te komen, wordt de kleurenrand gebruikt. Voor elke mogelijke combinatie van groene en blauwe linker- en bovenrand wordt er nog een aantal graden (0, 90, 180 of 270) bij opgeteld. Zo bekomt het algoritme de juiste oriëntatie. Doordat het geen enkele loop gebruikt en de hoeken al gedetecteerd zijn gebeurd dit in $O(1).