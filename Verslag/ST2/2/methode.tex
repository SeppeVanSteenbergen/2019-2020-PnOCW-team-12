Het onderzoek is volledig geautomatiseerd geweest aan de hand van Python. Binnen Python is gebruikt gemaakt van de bibliotheek OpenCv HIER MOET OOK NOG CITE KOMEN om de filters toe te passen op de afbeeldingen. De testen zijn uitgevoerd geweest op een gegevensbank met 670 afbeeldingen van schermen. Net zoals in het vorige onderzoek is gebruikgemaakt van verschillende omgevingsfactoren. Deze zijn echter buiten beschouwing gelaten in het onderzoek. 

Om de kwalitatieve data te kwantificeren is gewerkt met een scoresysteem. Een foto wordt voordat deze geanalyseerd wordt, onderverdeeld in een aantal blokken. Daarna worden binnen elk blok tien procent van de pixels willekeurig gekozen. Het onderverdelen in blokken zorgt ervoor dat de pixels mooi verspreid liggen binnen de afbeelding. Daarna worden de verschillende methodes op de foto toegepast. Om uiteindelijk een score aan de afbeeldingen toe te wijzen worden enkel de eerder willekeurig bepaalde pixels bekeken. Deze blijven wel dezelfde voor verschillende filters. De score geeft weer hoe dicht de kleuren van de samples bij de pure kleur ligt, met 0 volledig de andere kant van het kleurspectrum en 1 de pure kleur zelf.