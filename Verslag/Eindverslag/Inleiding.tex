\section{Inleiding}
De wereld wordt alsmaar digitaler. Mensen ontmoeten elkaar digitaal, spreken elkaar digitaal, zien elkaar digitaal. Ook het buitenspelen lijkt verleden tijd, samen een spelletje spelen online is ook geen rariteit meer. De digitale wereld past zich in sneltempo aan en zo moeten de toepassingen meegroeien. Wanneer mensen elkaar online willen zien of online een spel tegen elkaar willen spelen, willen ze dit het liefst zo echt mogelijk meemaken. Dit door een zo groot mogelijk scherm. Niet iedereen beschikt echter over de capiciteiten om een 100'' scherm aan te kopen. Een mogelijkheid die in dit verslag wordt besproken is het doen samenwerken van schermen. Eender welk scherm met een internetverbinding zou zo aan elkaar gekoppeld kunnen worden. 

Het verslag gaat eerst dieper in op het framework van de online applicatie, hoe master en clients communiceren, beveiliging... Vervolgens zal de synchronisatie van de schermen aan bod komen. Hierna wordt de detectie van de schermen besproken, de schermen zullen gedetecteerd worden door één enkele foto. Wanneer schermen overlappen zullen er hoeken niet zichtbaar zijn, het scherm moet gereconstrueerd worden, dit wordt besproken in deel \ref{sec:reconstructie}. Hierna komt nog de identificatie, waar staat elk scherm en de transformatie van de af te beelden objecten aan bod. We eindigen met triangulatie en vervolgens een kleine animatie die op de schermen al afgebeeld kan worden, dit om te tonen dat de schermen juist met elkaar verbonden zijn en er een vloeiende overgang mogelijk is.