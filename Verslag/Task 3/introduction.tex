Het herkennen van schermen, deze identificeren, lokaliseren uit een foto. Dit verslag behandeld deze uitdagingen. Het zoeken van schermen begint bij een foto. Deze foto bevat een scherm met een bepaald uitzicht, er is gekozen voor een groen-blauwe rand waarop gefilterd kan worden. Vervolgens zoekt het algoritme in de gefilterde foto naar aparte schermen en geeft aan hen een id. Met behulp van het kleurenverschil en bepaalde hoeken wordt de oriëntatie bepaald.

Het identificeren van het scherm gebeurd op dit moment nog apart met een kleurenbarcode. Deze barcode bevat 5 verschillende kleuren waardoor er 120 verschillende schermen geïdentificeerd kunnen worden. In de volgende weken worden deze twee, lokaliseren en identificeren, bij elkaar gezet.
\\
Dit verslag behandeld de keuzes die gemaakt zijn alsook een uitleg bij de gebruikte algoritmen en hun tijdscomplexiteit. De mogelijke beperkingen worden met deze kennis geduid.