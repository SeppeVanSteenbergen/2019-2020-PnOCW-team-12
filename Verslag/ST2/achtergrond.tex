\paragraph{RGB} is een additief kleurmodel waarbij een kleur wordt beschreven aan de hand van de drie primaire kleuren: rood, groen en blauw. Elke kleur wordt gevormd aan de hand van een combinatie van deze drie kleuren. RGB wordt heel veel gebruikt in grafische toepassingen. Wanneer een foto door een computer wordt uitgelezen zal dit ook in RGB-waarden gebeuren. Alleen zal dit model niet gebruikt kunnen worden voor detectie aangezien het model een niet-lineaire en discontinue ruimte vormt. Deze discontinuïteit maakt het beschrijven van een verandering in tint moeilijk. Daarnaast is het RGB-model gevoelig aan verandering van licht, wat resulteert in een verandering van tint.

\paragraph{HSL en HSV} maken allebei deel uit van het cylindrisch model. Deze worden beschreven in drie dimensies: een hoek, die de tint voorstelt gaande van 0\degree (rood), naar 120\degree (groen), richting 240\degree (blauw), om uiteindelijk bij 360\degree (rood) rond te zijn (zie figuur \ref{colorWheel}). Een horizontale dimensie, die de saturatie beschrijft en een verticale dimensie, die de lichtheid (HSL) of waarde (HSV) bepaalt. Het voornaamste voordeel en de reden voor het gebruik van deze modellen over RGB is het feit dat deze modellen imuun zijn aan veranderingen in licht. Want deze zitten in een aparte dimensie. Een ander voordeel is de continue tint in de HSV- en HSL-modellen. HSL wordt uiteindelijk boven HSV verkozen door zijn symmetrie voor licht en donker. \cite{inbook} \cite{rasouli2017effect}

\begin{figure}[h]
	\center
	\includegraphics[width=0.6\textwidth]{img/hslColorWheel.png}
	\caption{De waarden van de verschillende tinten in HSL. \cite{hslColorWheel}}
	\label{colorWheel}
\end{figure}