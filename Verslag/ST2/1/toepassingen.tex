Voor het detectiealgoritme kan het best gebruik gemaakt worden van de HSL kleurruimte zoals nu al het geval is. Voor het identificatiealgoritme waarvoor wit en zwart gebruikt wordt, zou beter overgeschakeld worden naar een algoritme dat gebruik maakt van de RGB kleurruimte. Ook is het beter om bij deze twee kleuren te kijken naar contrastovergangen. Omdat binnen HSL enkel gekeken werd naar de S- en L-waarde indien zwart of wit gedetecteerd werd, zullen deze twee parameters achterwege gelaten worden. Om kleuren te detecteren zal dus enkel nog rekening gehouden worden met de H-waarden.

De keuze voor de drie kleuren die weergegeven worden op het detectiescherm valt op rood, groen en blauw. Deze drie kleuren hebben een grote detectieratio en de omgevingsfactoren hebben het minste invloed op deze kleuren. In de praktijk was dit al als beste kleurencombinatie naar voren gekomen. Dit onderzoek heeft dit alleen maar bevestigd.

Aan de hand van de bevindingen omtrent de omgevingsfactoren, is gebleken dat de foto's zo dicht mogelijk bij de opstelling moeten getrokken worden. Dit zal verzekeren dat de omgeving minimaal is. Natuurlijk moet de hele opstelling wel binnen de foto passen.

