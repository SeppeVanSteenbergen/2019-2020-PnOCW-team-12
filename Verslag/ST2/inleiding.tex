Het vorige verslag behandelde de kleurdetectie, met focus op de te herkennen kleur en de invloeden van omgevingsfactoren. Bij het herkennen van kleuren spelen oneffenheden in de foto's ook een rol, nu zal hiernaar verder onderzoek gedaan worden. Een foto is onderhevig aan verschillende factoren, zo zal een artificiële intelligentie die vaak aanwezig is in een camera de foto al bewerken. Lichtinval en een ander scherm of zelfs vuil op dit scherm maakt dat de kleuren niet egaal worden weergegeven.

In sectie \ref{sec:achtergrond} komen verscheidene filters aan bod die de ruwheid in een foto kunnen reduceren, op welke manier doen ze dit, hoe zijn deze geïmplementeerd... Vervolgens wordt de toegepaste methode uitgediept, waarna er meer uitleg geveven wordt over de bevindingen van dit onderzoek. Een eerste zeer belangrijke is de tijdsconsumptie van de filter. Weegt de toeneming van uitvoeringstijd op tegen de verbetering van het resultaat? Daarnaast kan gekeken worden naar foto's die initieel niet correct werden gedetecteerd. Zorgt het toepassen van een filter ervoor dat deze wel correct wordt gedetecteerd? Welke filter heeft dan het beste resultaat? Zorgt het aanpassen van de kernel grootte voor betere detectie, en heeft dit effect op de uitvoeringstijd?  Ten slotte worden de belangrijke zaken opgesomt die kunnen bijdragen om het project te verbeteren.