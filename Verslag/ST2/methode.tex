Voor het onderzoek zijn 689 foto's genomen met drie verschillende smartphones. De gekleurde schermen werden weergegeven op drie verschillende laptops. Er is verder geen onderscheid gemaakt tussen de diverse modellen van schermen en smartphones. Het experiment kan dus uitgevoerd worden met om het even welke laptop en smartphone. De gebruikte kleuren en hun waarden binnen de onderzochte kleurruimten zijn weergegeven in tabel \ref{tab:kleuren}. De schermen zijn gefotografeerd onder variërende omstandigheden. Er is een onderscheid gemaakt tussen de hoeveelheid omgeving die meegenomen is in de afbeelding, onderverdeeld in vier categoriën. Elke kleur is gefotografeerd geweest met geen, 10\%, 40\% en 80\% omgeving. Daarnaast werd ook de helderheid van de schermen gewijzigd. De gebruikte standen zijn 25\%, 50\%, 75\% en 100\%. Als laatste factor zijn de foto's eens genomen in het donker, dus in afwezigheid van lichtbronnen maar ook eens met artificieel licht.  Tijdens de analyse van de afbeelding is dan nog eens onderscheid gemaakt tussen de twee kleurruimten, dit resulteert in een gegevensbank van 1378 afbeeldingen.  Meerdere componenten zijn nodig om kleuren te vergelijken. Een gegevensbank opgesteld met \textit{MySQL} en \textit{phpMyAdmin} houdt deze waarden bij. Het onderzoek baseert zich dan op deze verschillende eigenschappen. Als eerste wordt voor elke foto de kleur bijgehouden die weergegeven was op het scherm. Daarnaast is ook de kleur die het programma gedetecteerd heeft van belang. Voor deze twee kleuren wordt ook bijgehouden hoeveel procent van de foto uit die kleur bestaat. Zowel voor HSL als voor RGB kijkt het algoritme naar de kleurafstand. Bij RGB is dit de Euclidische afstand. 
$$ \mid RGB \mid = \sqrt{(R_1 - R_2)^2 + (G_1 - G_2)^2 + (B_1 - B_2)^2}$$
De methode vergelijkt zo elke pixel met de theoretische waarden uit tabel \ref{tab:kleuren}. De gedetecteerde kleur is diegene met de kleinste afstand tot de pixel. Voor HSL is ook de Euclidische afstand gebruikt. Echter kijkt deze enkel naar de hue-waarde, voor zwart en wit kijkt het algoritme naar de lichtsterkte. Minder dan tien procent wordt als zwart beschouwd, meer dan negentig als zwart. 
$$ \mid HSL \mid = \sqrt{(H_1 - H_2)^2 }$$
Ook de eerder vermelde factoren als omgeving en dergelijke worden mee opgeslagen. Als laatste geeft een boolean nog weer of de kleur effectief correct werd gedetecteerd.

\begin{center}
\begin{table}
\centering
\begin{tabular}{ | l | c | c | }
\hline
Kleur & [R, G, B] & Hue \\
\hline
Rood & [255, 0, 0] & 0 en 360 \\
Groen & [0, 255, 0] & 120 \\
Blauw & [0, 0, 255] & 240 \\
Geel & [128, 128, 0] & 60 \\
Cyaan & [0, 128, 128] & 180 \\
Magenta & [128, 0, 128] & 300 \\
Wit & [255, 255, 255] & $L <= 10$ \\
Zwart & [0, 0, 0]] & $L >= 90$ \\
\hline
\end{tabular}
\caption{De RGB en hue waarden van de theoretische kleuren.}
\label{tab:kleuren}
\end{table}
\end{center}