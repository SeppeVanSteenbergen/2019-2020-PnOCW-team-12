\documentclass[a4paper,11pt]{article}

\usepackage[margin=3cm]{geometry}

\usepackage{graphicx}
\usepackage{subcaption}
\usepackage[colorlinks,allcolors=violet]{hyperref}
\usepackage{url}
\usepackage{lmodern}


% https://tex.stackexchange.com/questions/94032/fancy-tables-in-latex
\usepackage[table]{xcolor}
\usepackage{booktabs}

\usepackage[utf8]{inputenc}

% https://tex.stackexchange.com/questions/664/why-should-i-use-usepackaget1fontenc
\usepackage[T1]{fontenc}
\usepackage{microtype} % good font tricks

\newcommand{\note}[1]{{\colorbox{yellow!40!white}{#1}}}
\newcommand{\exampletext}[1]{{\color{blue!60!black}#1}}

\begin{document}

\noindent
\colorbox[HTML]{52BDEC}{\bfseries\parbox{\textwidth}{\centering\large
  --- Code review P\&O CW 2019--2020 Task 4\&5 ---
}}
\\[-1mm]
\colorbox[HTML]{00407A}{\bfseries\color{white}\parbox{\textwidth}{
  Department of Computer Science -- KU Leuven
  \hfill
  \today
}}
\\

\smallskip

\noindent
%\mbox{}\hfill
\begin{tabular}{*4l}
\toprule
\multicolumn{3}{l}{\large\textbf{Team 12}} \\
\midrule
Martijn Debeuf &  &  \\ % fill in the time spend on this task per team member who worked on it and the amount of lines of code (LOC) reviewed
Frédéric Blondeel &  &  \\
\bottomrule
\hline
\end{tabular}\\
\\
Demo: \url{https://penocw.cs.kotnet.kuleuven.be:8012} \\
Files reviewed: delaunayTriangulation.js/Triangle.js

\noindent
{\color[HTML]{52BDEC} \rule{\linewidth}{1mm} }

\smallskip

\section{Delaunay Triangulation}
\subsection{delaunayTriangulation.js}
Er werd gekozen om een nieuwere syntax van javascript te gebruiken, de ";" is achterwege gelaten. 
Best wordt hier een algemene keuze in gemaakt die dan doorheen het hele project gebruikt wordt. In andere klassen 
is de oude syntax nog gehanteerd.
\begin{itemize}
\item In de statische functie "triangulation" wordt ook nog een lijst dezelfde naam gegeven als de functie. Beter wordt een andere naam gekozen om verwarring te vermijden.
\item Test functies werden onderaan het document geplaatst, mooier zou zijn moesten deze in aparte file staan.
\end{itemize}

\subsection{Triangle.js}
\begin{itemize}
	\item De if-statement lijn 19 kan vereenvoudigd worden door te schrijven:
	\begin{verbatim}
	return edge1.includes(edge2[0]) && edge1.includes(edge2[1])
	\end{verbatim}
\end{itemize}





\section{Advanced Screendetection}

\begin{itemize}
\item
\end{itemize}




\end{document}