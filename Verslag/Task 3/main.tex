\documentclass[a4paper,11pt]{article}

\usepackage[margin=3cm]{geometry}

\usepackage{graphicx}
\usepackage{subcaption}
\usepackage[colorlinks,allcolors=violet]{hyperref}
\usepackage{url}
\usepackage{lmodern}
\usepackage[dutch]{babel}

% https://tex.stackexchange.com/questions/94032/fancy-tables-in-latex
\usepackage[table]{xcolor}
\usepackage{booktabs}

\usepackage[utf8]{inputenc}

% https://tex.stackexchange.com/questions/664/why-should-i-use-usepackaget1fontenc
\usepackage[T1]{fontenc}
\usepackage{microtype} % good font tricks

\newcommand{\note}[1]{{\colorbox{yellow!40!white}{#1}}}
\newcommand{\exampletext}[1]{{\color{blue!60!black}#1}}

\begin{document}

\noindent
\colorbox[HTML]{52BDEC}{\bfseries\parbox{\textwidth}{\centering\large
  --- Verslag P\&O CW 2019--2020 Taak 3 ---
}}
\\[-1mm]
\colorbox[HTML]{00407A}{\bfseries\color{white}\parbox{\textwidth}{
  Department Computerwetenschappen -- KU Leuven
  \hfill
  \today
}}
\\

\smallskip

\noindent
%\mbox{}\hfill
\begin{tabular}{*4l}
\toprule
\multicolumn{2}{l}{\large\textbf{Team 12}} \\
\midrule
Frédéric Blondeel & 20h \\
Martijn Debeuf & 45h \\
Toon Sauvillers & 60h \\ % fill in the time spend on this task per team member who worked on it
Dirk Vanbeveren & 12h \\
Bert Van den Bosch & 52h \\
Seppe Van Steenbergen & 54h \\


\bottomrule
\hline
\end{tabular}\\

\noindent
{\color[HTML]{52BDEC} \rule{\linewidth}{1mm} }

\section{Introductie}\label{sec:introductie}
Het synchroniseren van de slave devices is essentieel om video of animatie op een correcte manier over meerdere schermen weer te geven. Het eerste deel van de opgave kijkt naar het absolute verschil tussen cliënt- en referentietijd. Doordat de gebruikte server gesynchroniseerd is met een atomische wereldklok, kan het absolute verschil in tijd worden gemeten. De testen gebruiken verscheidene besturingssystemen, aparaten en browsers. Het verslag licht ook het verschil tussen UDP en TCP, twee verbindingprotocols, toe.

\section{Algoritmen}\label{sec:technical-analysis}


\subsection{Scherm identificatie met barcodes}

\subsubsection{Algemene uitleg}

Voor de verschillende schermen te identificeren wordt gebruik gemaakt van een kleuren barcode. De barcode bestaat uit een herhalend patroon van 5 unieke kleuren telkens gevolgd door een witte lijn. Door het gebruik van deze witte lijn weet het algoritme waar het patroon eindigt en de volgende sequentie terug opnieuw begint. De 5 kleuren zijn speciaal gekozen om zo ver mogelijk van elkaar verwijderd te zijn in het HSL spectrum. Op deze manier is de kans op foute detectie zo klein mogelijk gemaakt. Voor de identificatie van de slaves wordt er dus een unieke combinatie van deze 5 kleuren weergegeven. Deze vormt dan een unieke vijfcijferige code die gelinkt kan worden aan de bijhorende slave. Dit zorgt ervoor dat we in theorie een totaal van $5!$ ($=120$) verschillende schermen op één moment kunnen detecteren. Herhaling van het patroon heeft als resultaat dat bij overlap het scherm nog steeds geïdentificeerd kan worden. Het algoritme zal van links naar rechts over de pixels, die zich in het midden van het scherm bevinden, itereren (complexiteit N). Vervolgens worden de HSL waarden van deze pixels bekeken om de overeenkomstige kleur van elke pixel te achterhalen. Wanneer een HSL waarde binnen de gewenste range valt, wordt het overeenkomstig cijfer opgeslaan in een lijst. Bij het bereiken van een witte lijn weet het algoritme dat het aan het einde van het patroon is. Wanneer dit het geval is wordt het inlezen van de volledige vijfcijferige code gecontroleerd op volledigheid. Zo niet wordt de lijst leeg gehaald en zoekt het algoritme verder. Het herhalend patroon is dus essentieel aan het correct inlezen van de barcode. Bijgevolg werd het aantal herhalingen van het patroon bekeken in een aantal testen. Uit deze testen werd een aantal van 7 herhalingen als optimaal resultaat bekomen. Een groter aantal herhalingen stemt overeen met meer kansen op een mogelijke detectie, maar stemt ook overeen met een kleinere oppervlakte per herhaling. Deze kleinere oppervlakte is dan weer nadelig voor detectie. 

\begin{figure}[h]
\centering
\includegraphics[scale=0.1]{barcode}
\caption{Voorbeeld van een barcode met 6 herhalingen.} 
\label{fig:barcode}
\end{figure}



\subsubsection{Data set}

Tijdens de testfase van het algoritme werden een aantal foto's genomen. Deze foto's verschilden in afstand tot het scherm, het scherm zelf (kleuren kunnen afwijken van scherm tot scherm) en het aantal herhalingen van het patroon (barcode).

\subsubsection{Experiments}\label{sec:experimenten}

\subsection{Schermherkenning}
Volgende algoritmen zijn toegepast op de input image file volgens uitgeschreven volgorde. Zo worden de schermen herkend.
\paragraph{Blauw en groen masker}
Uit de hsl image data matrix wordt een selectie van pixels gemaakt aan de hand van de doorgegeven parameters die als threshold gebruikt worden. De pixels die aan deze voorwaarden voldoen, worden opgeslagen in een matrix die maar één enkelvoudige waarde per pixel bevat. Dit resulteert in twee binaire matrices; voor een groene- en een blauwe threshold. 
\paragraph{Maskers combineren}
Het groene en blauwe masker vormen samen de border van een scherm, maar moeten nog samengevoegd worden door een envoudige matrix optelling van beide maskers.
Het maskeren en samenvoegen van deze maskers gebeurd telkens één iteratie door de volledige image matrix. Het totale maskeer proces gebeurd dan ook in $O(m*n)$ tijdscomplexiteit.

\paragraph{Median blur}
https://docs.opencv.org
Onze implementatie van median blur is gebasseerd op de documentatie van OpenCV. Elke pixel wordt vervangen door de mediaan van zijn geburige pixels. De range van deze buren wordt bepaald van de kernel size parameter k. Het algoritme gaat voor elke pixel in de matrix van de afbeelding nog eens door een subset van grootte k op k van deze matrix om een nieuwe waarde voor deze pixel te bepalen. Dit leidt tot een complexiteit van $O(k^2*m*2)$ met k de kernel size parameter en m en n de dimensie van de afbeelding.

\paragraph{Find islands}
De eerste implementatie was gebaseerd op een blob detection algoritme door The Coding Train, maar dit algoritme verreiste te veel manuele thresholds en was dus niet universeel genoeg voor te veel verschillende gevallen. Als bijvoorbeeld de resolutie van de foto niet groot was, dan moest er een hogere ‘afstandsthreshold’ gebruikt worden, maar dit gaf als bijwerking dat er soms meerdere schermen als enkel scherm werden aanschouwd.
Onze huidige implementatie is gebaseerd op een standaard 4-way floodfill algoritme met gebruik van een array als stack. Deze methode zorgt voor een veel nauwkeuriger resultaat dan het vorige algoritme. Vanaf dat er een witte pixel gevonden wordt, zal de selectie zich blijven uitbreiden via alle verbonden witte pixels. Het geïmplementeerde algoritme breidt zich uit volgens zijn 4 aanliggende buren en zo zal dus elke witte pixel maximaal vier maal in de stack terecht komen. In het slechtste geval is heel de foto wit en zal dit algoritme zich gedragen volgens $O(4mn)=O(mn)$. Maar dit zal gemiddeld veel lager zijn. Deze implementatie houdt net zoals in het blob detection algoritme een bounding box bij van de gevonden witte eilanden.

\paragraph{Find corners}
Find corners wordt uitgevoerd per gevonden eiland van het vorig algoritme. Aangezien er in deze fase nog gewerkt wordt met niet overlappende schermen, zullen de hoekpunten van een vierhoek altijd de grenzen aan zijn bounding box. Onze implementatie gaat dan ook de rand van de gevonden boundingbox (zie island detection) af om de vier hoekpunten te te bepalen.




\section{Testen}

Schrijfhulp Academisch Nederlands
Gerealiseerd door het Instituut voor Levende Talen
Tekst
controleren
Tekst
verrijken

STRUCTUUR EN SAMENHANG
Tekstanalyse
Verwijswoorden
Signaal- en structuurwoorden
Frequente inhoudswoorden
Vaak terugkerende patronen
Zinslengte
Alinealengte

STIJL
Formele en archaïsche woorden
Passieven
Omslachtig taalgebruik
Naamwoordstijl
Vage en algemene woorden
Persoonlijk taalgebruik
Informele, spreektalige en gekleurde woorden

SPELLING
Spelfouten
Aaneenschrijven of niet?
Afkortingen


Gebruiker afmelden

Welkom op de website van de Schrijfhulp Nederlands. De schrijfhulp werd ontwikkeld om studenten te ondersteunen tijdens het schrijfproces.

De Schrijfhulp Nederlands wil je bewustmaken van mogelijke foutenpatronen in de tekst. De aangeduide woorden of woordcombinaties in de tekst zijn dus niet noodzakelijk fout. Omdat de schrijfhulp zich focust op de drie niveaus (structuur en samenhang, stijl en spelling) worden niet alle mogelijke schrijfproblemen aangeduid.

Aantal (tussen)titels in deze tekst:
Deze informatie is nodig om correcte statistieken te kunnen berekenen.
(Tussen)titels moeten van alinea’s gescheiden zijn door een 'harde' enter. Klik hier voor meer uitleg.

tekst verwerkt!

\subsection{Animatie met timers}

Een animatie wordt smooth bevonden als er geen frame ontbreekt tijdens het afspelen. Hoe meer frames er per seconde afgebeeld kunnen worden, hoe smoother de animatie oogt. Het gemiddelde consumer-device heeft een refresh rate van 60Hz wat overeenstemt met 60 frames per seconde (fps). Om een animatie zo smooth mogelijk te laten afspelen, wordt er dus over het algemeen gemikt naar een afspeelsnelheid van 60fps.

Om elke seconde 60 frames af te beelden moeten de frames volgens een correct getimed interval de volgende frame afbeelden. In javascript zijn hier twee build-in functies voor beschikbaar, \texttt{setTimeout()} en \texttt{setInterval()}. Op deze manier kan er met een timer om de 16.7ms een nieuwe frame afgebeeld worden om de 60fps playback snelheid te behalen.

\subsection{Problemen met timers}
Bij het gebruik maken van timers wordt er verondersteld van een playback van 60fps omdat de meeste schermen dit hanteren, maar als er een trager scherm gebruikt wordt om deze animatie af te beelden zullen er overtollige frames ingeladen worden niet gerenderd zullen worden door de browser. Een laadtijd van 16.7ms of minder voor elke frame is ook niet gegarandeerd en wordt verder besproken in \ref{performance}.

\paragraph{Drift in timers}Deze timers zouden er voor zorgen dat er elke gerenderde frame door de browser, de juiste frame van de animatie bevat. Dit kan niet aangenomen worden aangezien er altijd een kleine fout zal bestaan op de wachttijd tot de executie van de opgeroepen functie van de timer. Bij het uitvoeren van een test gebaseerd op \cite{testDrift} valt te constateren dat veel populaire browsers de kleine fout niet compenseren en daardoor begint op te tellen en als gevolg de globale timing van de animatie volledig zijn synchronisatie verliest.

\textbf{MAYBE MEER BROWSERS?}

\begin{figure} [H]
	\centering
	\includegraphics [width=0.7\textwidth] {img/Timer drift.png}
	\caption{Timer drift van verschillende populaire browsers} \label{drift}
\end{figure}

Door het continu opschuiven van frame updates kan er een framedrop voorkomen door redundante stappen van de animatie in te laden tijdens de foute frame waardoor de animatie stappen verliest en haperend ervaren wordt.

\paragraph{Single threaded}
Javascript wordt single-threaded in de browser uitgevoerd. Door deze limitatie kan een taak die getimed staat over x aantal milliseconden pas uitgevoerd worden op de main thread als de event loop op dat moment beschikbaar is om die functie dan uit te voeren. Als er tussen het initialiseren van de timer en het uitvoeren van deze geplande taak een andere taak tussenbeide komt die langere executietijd heeft dan de geplande tijd zal de uitvoering uitgesteld worden tot de event loop terug vrij komt. Ook dit is een deuk in de betrouwbaarheid van timers en kan voor framedrops zorgen als de executie te lang wordt uitgesteld tot voor de nieuwe repaint.

\begin{figure} [H]
	\centering
	\includegraphics [width=0.6\textwidth] {img/drift-framedrop.png}
	\caption{Framedrop door gebruik van setTimeout} \label{drift-framedrop}
	\source{https://developers.google.com/web/fundamentals/performance/rendering/optimize-javascript-execution}
\end{figure}

\subsection{RequestAnimationFrame}

Een methode die moderne browsers bieden om animaties te updaten met javascript is \texttt{requestAnimationFrame()}. Het wordt in dezelfde context gebruikt als \texttt{setTimeout()}, maar lost enkele cruciale problemen op. Ten eerste moet er geen tijdsinterval meer gespecifieerd worden als parameter, maar synced de browser de executie van deze taak met de refreshrate van de monitor. De opgegeven callback functie wordt door de event loop net voor de repaint van de pagina uitgevoerd. Hierdoor worden enkel frames geüpdatet voor een repaint en zullen er geen redundante taken uitgevoerd worden per frame of framedrops voorkomen. \cite{requestFrameDocs}

\subsection{Performance} \label{performance}
Deze methode lost een groot deel van globale timing op, maar garandeert daarom geen smooth animation playback. We gaan er verder vanuit dat er geen devices een hogere refresh rate dan 60Hz verwachten en mikken dus op een playback snelheid van 60fps. Als de methode om de frame te updaten langer duurt dan 16.7ms zal de frame niet gedropt worden, maar uitgetrokken. Alle frames worden wel weergegeven, maar niet aan een snelheid van 60fps wat dus ook oogt als een niet vloeiende animatie.

\textbf{TODO}: performance bekijken met artif. workload (busy wait) en fps.

Om een pagina zo responsief mogelijk te houden, moet het aantal reflows geminimaliseerd worden. Er zijn nog verscheidene elementen die de reflow snelheid beïnvloeden zoals de diepte en afhankelijkheden van nodes in de DOM-tree van de webpagina. Aangezien er in dit geval enkel gewerkt wordt met een enkelvoudige animatie op een webpagina gaan we hier niet verder op in.
\cite{improvePerformance}

\subsection{HTTP request frames}

Voor deze test werd gebruik gemaakt van een sprite sheet met 15 frames op (figuur \ref{sheet}) en een animatie snelheid van 30 fps.

\begin{figure} [H]
	\centering
	\includegraphics [scale=0.7] {img/charging.png}
	\caption{Sprite sheet met 15 frames} \label{sheet}
\end{figure}

Bij de eerste test werdt de hele sprite sheet per frame ingeladen. Dit is om het effect te creëren dat de server elke frame genereert en doorstuurt naar de cliënt. Dit bleek zeer inefficiënt te zijn doordat de tijd om de frame op te halen met ethernet rond de 140 ms en met draadloos internet rond de 250 ms ligt (figuur \ref{draadloos} en figuur \ref{ethernet}).

\begin{figure} [H]
	\centering
	\begin{subfigure} [b] {0.45\textwidth}
		\centering
		\includegraphics [width=\textwidth] {img/draadloos.png}
		\caption{HTTP request met draadloze verbinding} \label{draadloos}
	\end{subfigure}
	\begin{subfigure} [b] {0.45\textwidth}
		\centering
		\includegraphics [width=\textwidth] {img/ethernet.png}
		\caption{HTTP request met ethernet verbinding} \label{ethernet}
	\end{subfigure}
\end{figure}

Hoewel de frames asynchroon worden ingeladen (figuur \ref{inladen}) is de animatie alles behalve ideaal. Door lange wachttijden zijn er veel frame skips en door de vele http requests wordt de verbinding gereset om de server niet te overbelasten met 30 requests per seconde (figuur \ref{reset}).

\begin{figure} [H]
	\centering
	\includegraphics [scale=0.3] {img/inladen.png}
	\caption{Inladen van elke frame} \label{inladen}
\end{figure}

\begin{figure} [H]
	\centering
	\includegraphics [scale=0.7] {img/reset.png}
	\caption{Connection reset door teveel requests} \label{reset}
\end{figure}

Bij de tweede test werd gebruik gemaakt van een buffer. De sprite sheet werd om de 0.5 seconden ingeladen zodat elke frame tijd had om te displayen ($30 \frac{frames}{sec} \frac{1}{15 frames}=\frac{0.5}{sec} $). Dit creëert het effect dat de server 15 frames genereert en doorstuurt in 1 keer i.p.v. frame per frame. Dit is duidelijk een verbetering want de server krijgt 15 keer minder requests (figuur \ref{inladen2}) en er zijn geen frame skips meer. Na het inladen van de sheet zal de cliënt de 15 frames achter elkaar drawen zonder te hoeven wachten op de server. Daarna zal de volgende sheet (of buffer) gebruikt worden.

\begin{figure} [H]
	\centering
	\includegraphics [scale=0.7] {img/inladen2.png}
	\caption{Inladen van de buffer} \label{inladen2}
\end{figure}

\section{Besluit}\label{sec:besluit}
De basis functionaliteiten zijn al op hun plek en werken al aanzienlijk goed samen. Een belangrijk onderdeel voor een accurate projectie van de foto is de detectie van het scherm en het reconstrueren van overlappende en verborgen delen van een scherm. De huidige kleurenfilters voor de detectie gaan nog verder uitgewerkt worden aan de hand van een verbeterde threshold formule die meerde kanalen van het HSL--spectrum gebruikt. Ook de reconstructie methode gaat nog uitgebreid en verfijnd worden om niet enkel meer variërende situaties op te lossen, maar ook een nauwkeuriger resultaat te leveren.

\bibliographystyle{plain}
\bibliography{bibliography}



\end{document}