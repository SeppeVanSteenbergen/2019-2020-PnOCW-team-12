\documentclass[a4paper,11pt]{article}

\usepackage{listings}
\usepackage{tikz}
\usetikzlibrary{shapes.geometric, arrows}
\usepackage[margin=3cm]{geometry}
\usepackage{amsmath}
\usepackage{graphicx}
\usepackage{subcaption}
\usepackage[colorlinks,allcolors=violet]{hyperref}
\usepackage{url}
\usepackage{float}
\usepackage{lmodern}
\usepackage[dutch]{babel}
\usepackage{hyperref}
\usepackage{subcaption}
\usepackage{listings}
\usepackage{tikz}
\usetikzlibrary{shapes.geometric, arrows, positioning}
\usepackage{gensymb}


\colorlet{punct}{red!60!black}
\definecolor{background}{HTML}{EEEEEE}
\definecolor{delim}{RGB}{20,105,176}
\colorlet{numb}{magenta!60!black}
\lstdefinelanguage{json}{
basicstyle=\normalfont\ttfamily,
numbers=left,
numberstyle=\scriptsize,
stepnumber=1,
numbersep=8pt,
showstringspaces=false,
breaklines=true,
frame=lines,
backgroundcolor=\color{background},
literate=
*{0}{{{\color{numb}0}}}{1}
{1}{{{\color{numb}1}}}{1}
{2}{{{\color{numb}2}}}{1}
{3}{{{\color{numb}3}}}{1}
{4}{{{\color{numb}4}}}{1}
{5}{{{\color{numb}5}}}{1}
{6}{{{\color{numb}6}}}{1}
{7}{{{\color{numb}7}}}{1}
{8}{{{\color{numb}8}}}{1}
{9}{{{\color{numb}9}}}{1}
{:}{{{\color{punct}{:}}}}{1}
{,}{{{\color{punct}{,}}}}{1}
{\{}{{{\color{delim}{\{}}}}{1}
{\}}{{{\color{delim}{\}}}}}{1}
{[}{{{\color{delim}{[}}}}{1}
{]}{{{\color{delim}{]}}}}{1},
}


% https://tex.stackexchange.com/questions/94032/fancy-tables-in-latex
%\usepackage[table]{xcolor}
\usepackage{booktabs}

\usepackage[utf8]{inputenc}

% https://tex.stackexchange.com/questions/664/why-should-i-use-usepackaget1fontenc
\usepackage[T1]{fontenc}
\usepackage{microtype} % good font tricks

\newcommand{\note}[1]{{\colorbox{yellow!40!white}{#1}}}
\newcommand{\exampletext}[1]{{\color{blue!60!black}#1}}

\begin{document}

\noindent
\colorbox[HTML]{52BDEC}{\bfseries\parbox{\textwidth}{\centering\large
--- Eindverslag P\&O CW 2019--2020 ---
}}
\\[-1mm]
\colorbox[HTML]{00407A}{\bfseries\color{white}\parbox{\textwidth}{
Department Computerwetenschappen -- KU Leuven
\hfill
\today
}}
\\

\smallskip

\noindent

\begin{tabular}{*4l}
\toprule
\multicolumn{2}{l}{\large\textbf{Team 12}} \\
\midrule
Frédéric Blondeel &\\
Martijn Debeuf &\\
Toon Sauvillers &\\ % fill in the time spend on this task per team member who worked on it
Dirk Vanbeveren  &\\
Bert Van den Bosch & \\
Seppe Van Steenbergen &\\


\bottomrule
\hline
\end{tabular}\\
\noindent
{\color[HTML]{52BDEC} \rule{\linewidth}{1mm} }


\section*{Abstract}\label{sec:abstract}
Het laten samenwerken van verschillende schermen van computers, tablets, smartphones en alle andere soorten, met toegang tot een moderne web-browser, om samen een beeld volledig beeld te vormen en dit alles geconfigureerd door één master device is het doel van deze bachelorproef. De devices zijn samen verbonden aan een web-applicatie en worden geconfigureerd door door een foto te nemen met het master device door de gebruiker. Met of zonder overlap, sommige dicht en sommige schermen verder van de camera, recht of gekanteld, het eindresultaat dat de schermen weergeven moet vlak tonen vanuit de configuratiepositie van de gebruiker.

Het start allemaal met het detecteren van de verschillende schermen. Voor ons was het vanaf het eerste moment het doel om het volledige resultaat te bereiken door het maken van één enkele foto door een standaard smartphone camera met alle gewenste schermen aanwezig. Dit zou het lastige probleem van een inconsistente camera positie van verschillende foto’s meteen elimineren. Al vanaf de eerste resultaten werkte we met voorbepaalde kleuren en filteren op geassocieerde color ranges en vormen die op de slaves worden afgebeeld voor de detectie en maskers van de te detecteren schermen te vormen. Met behulp van onderzoek naar HSL colorranges en floodfill worden de maskers en individuele associatie van de schermen afgeleid uit de foto. Verdere algoritmes voor hoekdetectie op elke scherm en afgedekte delen van de schermen te reconstrueren gebeuren op deze maskers. De link tussen de gevonden schermen en clients die verbonden zijn met de server via de web-applicatie worden gelinkt door het lezen van een unieke kleuren-barcode die overeen komt met een client-ID. Met deze overeenkomende informatie tussen client en gevonden scherm wordt er een perspectief matrix afgeleid om een correct resultaat op elk scherm af te beelden dat zowel de continuïteit en visuele vlakheid van het resultaat behoud en garandeerd.

Tijdens de eindfases en het verder drijven van de omstandigheden is het duidelijk geworden dat de kleurendetectie de grootste struikelblok was van de volledige methode. Deze aanpak viel al snel uit elkaar bij relatief kleine uitwijkingen in de omgevingsbelichting en reflecties in de schermen. Enkel bij schermen en camera’s die gebruikt werden voor het afstellen van de kleurdetectie kregen we een consistent resultaat. Bij het terugblikken is een makkelijke oplossing om de barcode voor indentificatie te beperken tot grijswaarden om zo een grotere range voor de kleurendetectie te laten. Zo worden er minder kleuren binnen het HSL spectrum door elkaar gehaald en is er meer ruimte binnen het spectrum voor elke kleur. De ontworpen algoritmes voor hoekdetectie en reconstructie werken bij een correct gevormd masker naar een verlangend resultaat.

\newpage

\tableofcontents
\newpage

\section{Inleiding}
De wereld wordt alsmaar digitaler. Mensen ontmoeten elkaar digitaal, spreken elkaar digitaal, zien elkaar digitaal. Ook het buitenspelen lijkt verleden tijd, samen een spelletje spelen online is ook geen rariteit meer. De digitale wereld past zich in sneltempo aan en zo moeten de toepassingen meegroeien. Wanneer mensen elkaar online willen zien of online een spel tegen elkaar willen spelen, willen ze dit het liefst zo echt mogelijk meemaken. Dit door een zo groot mogelijk scherm. Niet iedereen beschikt echter over de capiciteiten om een 100'' scherm aan te kopen. Een mogelijkheid die in dit verslag wordt besproken is het doen samenwerken van schermen. Eender welk scherm met een internetverbinding zou zo aan elkaar gekoppeld kunnen worden. 

Het verslag gaat eerst dieper in op het framework van de online applicatie, hoe master en clients communiceren, beveiliging... Vervolgens zal de synchronisatie van de schermen aan bod komen. Hierna wordt de detectie van de schermen besproken, de schermen zullen gedetecteerd worden door één enkele foto. Wanneer schermen overlappen zullen er hoeken niet zichtbaar zijn, het scherm moet gereconstrueerd worden, dit wordt besproken in deel \ref{sec:reconstructie}. Hierna komt nog de identificatie, waar staat elk scherm en de transformatie van de af te beelden objecten aan bod. We eindigen met triangulatie en vervolgens een kleine animatie die op de schermen al afgebeeld kan worden, dit om te tonen dat de schermen juist met elkaar verbonden zijn en er een vloeiende overgang mogelijk is.

\section{Framework}\label{sec:framework}
In dit project wordt gebruik gemaakt van het \textit{VueJS framework} voor de frontend. Hierdoor is het eenvoudiger om de code in verschillende bestanden te verdelen. Ook is het handig om hiermee de user interface te ontwerpen en implementeren. 


\subsection{Masters en clients}
Bij het openen van de webpagina is er een knop om zich eerst aan te melden bij het platform.
Dit zorgt ervoor dat later een login systeem met wachtwoord gemakkelijk kan worden toegevoegd indien dit gewenst is.
Hierna heeft men de keuze om master of client te worden, deze informatie wordt in het geheugen van de server opgeslaan. 

\paragraph{Kamers} Elke master heeft zijn eigen kamer waarbij clients zich kunnen aansluiten. Dit maakt het mogelijk de applicatie simultaan te gebruiken met meerdere masters. Telkens wanneer een commando wordt verstuurd krijgen enkel de clients in die specifieke kamer dat signaal.

\subsection{Communicatie protocollen}
%hoe en waarom de commando's voor het scherm worden verstuurd
Om de communicatie tussen master en clients gestructureerd te laten verlopen, worden verschillende manieren gebruikt.
Hieronder worden de verschillende gebruikte protocollen om data te verzenden uitgelegd.
\subsubsection{Commando's}
Elke client kan op de canvas van zijn scherm pijlen tekenen, de achtergrond van kleur laten veranderen of een afbeelding weergeven. De commando's om de schermen te controleren via de master worden allemaal doorgestuurd via \textit{SocketIO}. Deze techniek is gekozen over het \textit{http-protocol} omdat de hoeveelheid te verzenden data niet zo groot is en SocketIO ook op constante basis de client, master en server aangesloten houdt. Hierdoor worden alle berichten onmiddelijk ontvangen. Een voorbeeld van wat er wordt verstuurd word weergegevin in figuur \ref{jsonScreenCommand}.
Enkel het 'payload' gedeelte wordt verstuurd naar de clients vanaf de server.

\begin{figure} [h]
    \begin{lstlisting}[language=json,firstnumber=1]
    {s
    payload:{
    type: "flood-screen",
    data: {
    command: [{
    type: "color"/"interval",
    value: "[255,0,0]" / "200" }] (integer or rgb list)
    }
    }
    to: [user_id or "all"](to single user or all users in room)
    }

    \end{lstlisting}
    \caption{Voorbeeld van een JSON commando om schermen van kleur te laten veranderen, verzonden naar de server} \label{jsonScreenCommand}
\end{figure}

\subsubsection{Video verzenden}
De video's worden niet direct verstuurd naar elke client om af te spelen. Dit zou een lange wachttijd vereisen vooraleer het video bestand helemaal verzonden is en ontvangen door elke client.
Het uploaden van video is zo geïmplementeerd dat deze eerst naar de server wordt verstuurd via een HTTPS POST request (door middel van een HTML Form) en opgeslagen als een bestand op de server. Vervolgens wordt naar elke client de link doorsgestuurd vanwaar ze de video kunnen streamen via een HTML Video element. De video wordt dan gebufferd en kan onmiddelijk worden afgespeeld zonder dat het hele bestand al gedownload is.

\subsection{Beveiliging}
\paragraph{HTTPS} Staat voor Hypertext Transfer Protocol Secure. Door dit protocol zijn alle berichten geëncrypteerd waardoor het voor buitenstaanders niet zomaar mogelijk is verzonden berichten te lezen .
\paragraph{Login security} Naast HTTPS is het ook belangerijk dat een aangesloten gebruiker zich niet zomaar kan voordoen als een andere gebruiker. Of eender wie zomaar aan alle informatie kan. Om deze redenen hebben alle ingelogde gebruikers een geëncrypteerde \textit{cookie}] die hun gebruikers\_id bevat. Dit maakt het mogelijk om elke verschillende gebruiker te identificeren in de backend-server door deze te decrypteren. Hierdoor is het mogelijk om bepaalde server endpoints, met bijvoorbeeld video of foto bestanden, enkel aan ingelogde gebruikers bloot te stellen en niet aan andere gebruikers die geen inlog cookie hebben.
Vervolgens is veiligheid ook een probleem bij sockets. Als sockets hun aansluiting verliezen worden ze automatisch opnieuw aangesloten.Hierbij veranderd enkel hun id. Om te kunnen zien tot welke client een socket toebehoort, moet elke gebruiker zijn gebruikers\_id verzenden via de socket om deze aan zijn gebruikers\_id te kunnen linken. Deze informatie wordt allemaal in de server opgeslaan.

\subsection{Uitbreidingsmogelijkheden}
\paragraph{Login systeem} Momenteel is er een anoniem login systeem ingeboud waar er geen wachtwoord voor vereist is. Doordat dit nu al ingewerkt zit, kan dit later makkelijk omgevormd worden naar een login met inloggegevens.
\paragraph{Foto upload met https} Nu worden de foto's doorgestuurd via sockets. Later is het de bedoeling om net zoals bij video de afbeeldingen ook door te sturen via https. Zodat de master slechts één keer de foto moet uploaden en de clients dan zelf de foto aanvragen.
\paragraph{SocketIO beveiliging} Om nu de socket te identificeren moet elke gebruiker zijn gebruikers\_id versturen. Dit kan voor problemen zorgen als een mogelijke hacker een andere gebruikers\_id verzend via de socket om zich als iemand anders voor te doen. Om dit te voorkomen kan een tijdelijk inlog wachtwoord aangemaakt worden dat de gebruiker kan verzenden om zich te identificeren via de socket. 
 
 
 
 
 

\section{Synchronisatie}\label{sec:synchronisatie}
Bij een live applicatie is het belangrijk dat de aangesloten apparaten de mogelijkheid hebben synchroon een beeld weer te geven. Dit is niet altijd vanzelfsprekend dankzij verschillende factoren, zoals vertraging op en snelheid van een apparaat. Hiervoor zijn er verschillende technieken om toch synchronisatie te bekomen.

\subsection{Vertraging berekenen}
Er is een vertraging tussen een aangesloten client en de server, de {\it ping}. Dit is gemeten in miliseconden.
Deze wordt gemeten door een bericht met de actuele tijd te verzenden van de server naar de client, en terug. De verzonden tijd wordt afgetrokken van de actuele tijd waarmee de ping verkregen is.
In figuur \ref{diag} is de informatieoverdracht zichtbaar. In de server wordt de servertijd (TS1) berekent, verzonden naar de client en teruggekregen.
Nu wordt de actuele tijd berekent in de server TS2. Dus de uiteindelijke ping is:
\[ping = TS2 - TS1\]

\tikzstyle{client} = [rectangle, rounded corners, minimum width=3cm, minimum height=2cm,text centered, draw=black, fill=blue!30]


\begin{figure}
    \begin{tikzpicture}[node distance=1.5cm and 2cm]

        \node (A) at (3.5, 1.5){};
        \node (B) at (8,1.5){};
        \node (C) at (6,1.8){};
        \node[red] (D) at (10,0.4){};
        \node (S1) at (1.5cm,0.5cm){};
        \node (S1T) at (3.25cm, 0.7cm){TS1};
        \node (C2T) at (3.25cm, -0.3cm){TS1+TC};
        \node (C1) at (5cm,0.5cm){};
        \node (F) at (12,1.5){};
        \node (G) at (14,1.5){};
        \node (H) at (13,1.7){};

        \node (serv) [client] {Server};
        \node (clie) [client, right of=serv, xshift=5cm] {Client};

        \path[->]
        (S1) edge  (C1)
        (clie) edge (serv);

    \end{tikzpicture}
    \caption{Diagram van informatieoverdracht. (TS: Time Server, TC: Time Client)} \label{diag}
\end{figure}

\begin{figure}
    \begin{lstlisting}[language=json,firstnumber=1]
    {
    type: 'count-down',
    data: {
    start: [integer]
    interval: [integer, in ms],
    startTime: [date in ms]
    }
    }
    \end{lstlisting}
    \caption{countdown JSON commando verzonden naar clients} \label{json2}
\end{figure}

\subsection{Kloksynchronisatie}
Het is niet gegarandeerd dat de klokken van de clients allemaal gesynchroniseerd zijn met de server. Daarom is het ook nodig om te weten wat het verschil is tussen de tijd aan de kant van de client en de server.
Bij het terug verzenden van de client naar de server wordt de clienttijd (TC) bij het bericht gezet. Met deze TC en de berekende {\it ping}, is het mogelijk het tijdsverschil tussen de client en de server te bepalen ($DeltaTime$). Door dit verschil toe te voegen aan de servertijd is het mogelijk de correcte clienttijd te vinden. Dit wordt gebruikt om een starttijd te bepalen voor elke client dat op exact hetzelfde moment zal beginnen:
\[DeltaTime = (TC+ping/2) - TS2\]

Zo is de tijd van de client ten opzichte van de server altijd:
\[TimeClient = TimeServer + DeltaTime\]


\subsection{Aftelklok}
Bij de naïeve implementatie van de aftelklok is er gebruik gemaakt van de {\it setTimer()}-functie die recursief een getal aftelt en tekent op een canvas. Een probleem hierbij is dat apparaten niet even snel het getal kunnen tekenen op het scherm waardoor er apparaten kunnen zijn die sneller zijn dan anderen.
Daarom is er gebruik gemaakt van {\it setInterval()} dat periodisch het getal berekent, relatief ten opzichte van de meegegeven starttijd. Als er een client trager is en niet op tijd op het scherm kan tekenen, dan zal het getal worden overgeslagen omdat het apparaat het sowieso niet zou aankunnen. Hierdoor blijven de getallen op het scherm synchroon en zal het aftellen ook op hetzelfde moment stoppen op elk scherm.
Het getal is als volgt berekend:
\[number = startNum - Math.floor((actualTime - startTime) / interval)\]




\section{Algemeen verloop detectie algoritme}
Er gaat heel wat vooraf aan het tonen van een afbeelding op alle cliënts. In bijlage \ref{bijlageA1} is een algemeen overzicht te vinden.
Het begint allemaal bij de afbeelding met alle schermen op, deze zal geanaliseerd worden. Om beter met kleuren te werken, wordt de afbeelding van {\it RGBA} naar het {\it HSLA} spectrum worden gebracht. Vervolgens wordt er gefilterd op de randkleuren van het detectiescherm. Hieruit worden de zogenaamde {\it islands} gehaald die gefilterd worden tot er enkel mogelijke schermen overblijven.
\paragraph{Hoekdetectie}
Vervolgens zoekt het algoritme naar hoeken, zie bijlage \ref{bijlageA2}. Er wordt gekeken of het scherm gekanteld of recht staat en met deze info het bijpassende hoekdetectie algoritme toegepast. Hierna worden de hoeken gefilterd en zijn de hoeken gedetecteerd. Als er niet genoeg hoeken gevonden zijn, is het geen geldig {\it island}.

\paragraph{Hoekreconstructie}
In bijlage \ref{bijlageA3} en deel \ref{sec:reconstructie}, staat het hoekreconstructie gedeelte gedetailleerder beschreven. Dit wordt uitgevoerd wanneer er niet genoeg hoeken zijn herkend voor een scherm, maar wel genoeg om te reconstrueren. Het gaat de missende hoeken reconstrueren door de lijnen vanuit de reeds gevonden hoeken te volgen.
\paragraph{}
Met alle hoeken op zak is er een scherm gevonden en wordt het geïdentificeerd met behulp van een barcode. Het scherm wordt toegewezen aan een cliënt. Het zoeken gaat verder tot alle cliënts gevonden zijn of alle pixels uit de afbeelding zijn overlopen.



\section{Detectie}\label{sec:detectie}
\subsection{Flood fill}
Elke individuele slave screen laat een vooraf bepaalde afbeelding zien met gekende kleur-waarden. Om deze schermen te detecteren wordt de foto gefilterd op basis van gekende HSL ranges (TODO: verwijzing naar kleur van Seppe) en een standaard four-way flood fill algoritme [4 way afbeelding of te veel?] (*verwijzing wikipedia flood fill) om de associatie van de verbonden pixels te behouden. Na de executie van de flood fill is er voor elke gedetecteerd eiland een pixel masker met een unieke ID waar de verdere bewerkingen op uitgevoerd zullen worden. Het geimplementeerde flood fill algoritme groeit volgens de vier pixel-buren en een stack-based iteratie process om recursie-overflow tegen te gaan bij grote afbeeldingen en eilanden. In een worst-case scenario zal dit algoritme een eiland detecteren over de volledige afbeelding. Aangezien elke pixel maximaal vier keer in de stack terecht kan komen, door zijn vier buren, loopt deze flood fill volgens een tijdscomplexiteit van 
\[O(4mn)=O(mn)\]
 met m en n de dimensies van de afbeelding. De grootte van elk eiland zal in de praktijk over het algemeen een stuk kleiner zijn dan de volledige afbeelding (*verwijzing tijdscompl.).

\begin{figure}[h]
\centering
\includegraphics[scale=0.6]{img/mask.png}
\caption{Kleurenmasker van een scherm}
\end{figure}

Niet enkel de associatie van de gemaskeerde pixels wordt op deze manier behouden, maar deze methode maakt ook dat de komende bewerkingen maar over een minimale bounding box uitgevoerd worden ten opzichten van de volledige pixel matrix.
De afbeelding dat op elk slave-screen afgebeeld wordt voor detectie, is een combinatie van vorige iteraties van het project. Het combineren van een border met een kruis geeft het meeste informatie over de associatie van de kleur-gefilterde pixels bij overlap of afgedekte delen van het scherm en bovendien meer informatie over de oriëntatie van het scherm op de foto.

\subsection{Hoeken}
In de eerste stap wordt er bepaald of het scherm voornamelijk recht of gekanteld is ten opzichte van de foto. Hiervoor wordt langs de linker kant van het kader nagegaan hoe hoog de standaarddeviatie van witte pixels is. Bij een standaardafwijking onder de 15\% wordt een scherm als liggend of verticaal gezien op de foto.\\
Er werden vooraf algemene hoek-detectie algoritmes zoals shi-tomashi geïmplementeerd en getest, maar gaven een te complex resultaat op de binaire maskers om de juiste hoeken te filteren. Dit algoritme draagt ook een relatief grote overhead door de x- en y-sobel operaties die in een eerste stap toegepast moeten worden. Onze algoritmes zijn veel simplistischer, maar geven een perfect bruikbaar resultaat voor onze noden.\\
Als in eerste instantie het scherm als gedraaid beschouwd wordt, zal er vanuit elke rand van de bounding box van het eiland het de eerste mask-pixel als corner beschouwd worden. In het geval dat het scherm relatief horizontaal of verticaal recht staat, zal er loodrecht op de randen gezocht worden, maar volgens een diagonaal tot een mask-pixel [zie afbeeldingen].\\

\begin{figure}[h]
\centering
\includegraphics[scale=0.6]{img/perpSearch.png}
\caption{Loodrechte search}
\end{figure}

Beide variaties van hoek-detectie zal altijd vier hoeken als resultaat opleveren. Dit zullen door overlap en foutjes in het maskeren niet altijd correcte hoeken zijn. Na het bepalen van van de hoeken worden de hoeken nagekeken of deze resultaten wel degelijk kwalificeren als hoek. Deze kwalificatie is gebaseerd op de features die in de buurt van elke hoek gevonden moeten worden, namelijk twee lijnen die hoek van de border vormen en een diagonaal lijn naar het middelpunt van het scherm. Deze lijnen zijn bepaald door te filteren door de border- en diagonaal-kleur die gescheiden zijn door een witte rand. Als in  een eerder gevonden hoek deze features niet in zijn omgeving te vinden zijn, wordt deze hoek verworpen. 
[corner filter afbeeldingen]\\

\begin{figure}[h]
\centering
\begin{subfigure}{0.5\textwidth}
\centering
\includegraphics[width=0.6\textwidth]{img/correctCorner.png}
\caption{Een correcte hoek}
\end{subfigure}%
\begin{subfigure}{0.5\textwidth}
\centering
\includegraphics[width=0.6\textwidth]{img/notACorner.png}
\caption{Hoek zonder volledige features}
\end{subfigure}{}
\end{figure}

(linken/binden naar reconstruction) Als er geen vier correcte hoeken gevonden worden, zullen missende hoeken gereconstrueerd worden uit de wel gevonden hoeken en het middelpunt.

\section{Reconstructie}\label{sec:reconstructie}

	Na het filteren van de hoeken kan het dus voorkomen dat er geen vier meer overblijven. Volgende kunnen hiervoor de oorzaak zijn. Enerzijds is het mogelijk dat bepaalde delen van schermen elkaar overlappen in de opstelling, anderzijds kunnen één of meerdere hoeken niet of slecht detecteerbaar zijn door een obstakel. Er wordt opgelegd dat minstens twee aanliggende hoekpunten en het middelpunt zichtbaar zijn. Indien men zich aan deze vooropgestelde eis houdt kan met volgend algoritme het scherm steeds volledig gereconstrueerd worden. Na een opsomming van de stappen volgt een meer gedetailleerde uitleg.
	\paragraph{Algoritme}
	 De input die wordt meegegeven is een dictionary van de reeds gevonden hoekpunten. De sleutels zijn LU,RU,RD en LD m.a.w. posities van hoeken en als bijhorende waarden coördinaten voor deze die reeds gevonden zijn en een \textit{null} als plaatshouder voor de nog te reconstrueren hoeken. \newline
	 Eerst worden de vier punten bepaald die zich rond het middelpunt op de diagonalen bevinden. Deze worden ook in een dictionary opgeslagen met dezelfde structuur als de input. Daarna wordt hoek per hoek gekeken welke nog ontbreken. Indien reconstructie nodig is, wordt het overeenkomende punt van de diagonalen genomen. Samen met het middelpunt wordt hieruit een eerste reconstructielijn opgesteld. Daarna wordt vanuit een aanliggend hoekpunt het laatste hulppunt bepaald waardoor de tweede rechte wordt getrokken. De gezochte hoek is dan het snijpunt van de twee constructielijnen. Deze stappen herhalen voor andere ontbrekende hoeken resulteert in een dictionary met alle hoekenpunten van het scherm. Hieronder volgt een uitgebreidere uitleg van gebruikte methodes met veronderstellingen, voordelen en nadelen.
	
	\subsection{Hulppunten op diagonalen} \label{subsec:diagonalen}
		
		Telkens wanneer reconstructie nodig is zulllen de punten op diagonalen rond het middelpunt bepaald en geordend opgeslagen worden in een dictionary. Hiervoor worden alle pixels overlopen die op de cirkel met een bepaalde radius rond het middelpunt liggen overlopen. De straal van deze cirkel wordt gedefiniëerd als een vierde van de grootste afstand tussen de reeds gevonden hoeken en het middelpunt. Op deze manier wordt rekening gehouden met de grootte van het scherm. 25 procent van deze afstand nemen zorgt ervoor dat de pixels niet tot het middelpunt behoren en er zich tegelijkertijd niet te ver van bevinden. 
		\paragraph{Startpunt} Het startpunt van waaruit de cirkel doorlopen wordt, is een pixel die niet tot een diagonaal behoord. Waarom dit belangrijk is zal later duidelijk worden. Bepalen of een punt deel uitmaakt van een diagonaal gebeurt aan de hand van een functie die controleerd of er zich tussen de desbetreffende pixel en een tweede meegegeven punt (in dit geval het middelpunt) een wit deel pixels bevindt. Indien er tussen de twee meegegeven punten een stuk wit voorkomt, betekend dit dat de pixel afkomstig is van een deel barcode \ref{Foto van screendetectie html!!!!} (*verwijzing foto van screendetectie html).
		Doordat de kleuren van de boord en diagonalen ook in de barcode voorkomen, sluit deze functie dus eigenlijk uit dat stukjes barcode toegevoegd worden aan de lijst. Het interval waarmee de hoek van nul tot twee keer pi loopt is één procent, hiermee worden net genoeg pixels overlopen. Naar later toe zou deze waarde ook relatief kunnen gezet worden naar de grootte van het scherm. Voor elke aaneenschakeling van pixels die geen wit kruisen op hun pad naar het middelpunt wordt een lijst aangemaakt tijdens het doorlopen van de cirkel. Uiteindelijk heeft men dan vier sublijsten van pixels in een lijst die elk een diagonaal voorstellen. Uit elke lijst wordt dan het middelste element genomen die het midden van dat cirkelsegment is. Hier komt al een eerste voordeel van het bepaalde startpunt naar voor. Indien dit niet gedaan zou worden kon het voorkomen dat een cirkel binnen een diagonaal startte. Dit zou resulteren in meer dan vier lijsten die aangemaakt worden wat het bepalen van de middens aanzienlijk complexer zou maken. 
		\paragraph{Ordenen} Zoals eerder vermeld worden deze vier punten dan vanuit een lijst geordend in een dictionary geplaatst. Als referentie wordt gestart vanuit de HTML voor de schermdetectie. Daarin zijn de twee bovenste hoeken geel en de onderste roze. Door de manier waarop de punten bepaald werden (het doorlopen van een cirkel) zitten de gele en roze pixels alreeds bij elkaar. Dus moet enkel nog gecontroleerd worden of de twee gele punten de eerste twee elementen in de rij zijn. Is dit niet het geval wordt de lijst geroteerd totdat aan de voorwaarde is voldaan. Elk element wordt dan in die volgorde toegevoegd aan de dictionary waardoor de punten telkens dezelfde ordening zullen hebben.
	
	\subsection{Reconstrueren van een hoekpunt}
		
		Indien een hoekpunt moet gereconstrueerd worden met andere woorden de waarde van de desbetreffende sleutel in de dictionary van hoekpunten is \textit{null}, zijn vier punten nodig waaruit twee rechten kunnen worden opgesteld waarvan het snijpunt dan de nieuwe hoek is. 
		\paragraph{Eerste hulppunt} In de daarnet aangemaakte dictionary wordt het punt op de diagonaal geselecteerd die hoort bij het ontbrekende hoekpunt. Als bijvoorbeeld de bijhorende waarde van LU \textit{null} is, wordt in de dictionary van punten op de diagonalen ook de waarde van LU geselecteerd.
		 Vanuit dit punt en het middelpunt kan de vergelijking voor een rechte opgesteld worden die de eerste constructielijn zal vormen. 
		 Een aanliggend hoekpunt met gekende coördinaat, hulphoek genaamd, zal door de voorwaarde van minstens twee aanliggende detecteerbare hoeken steeds gevonden worden. 
		Vanuit deze worden weer de cirkelsegmenten van boorden en diagonaal bepaald, dit met de radius die berekend werd in \ref{subsec:diagonalen}. Met als grote verschil dat nu niet de middens moeten bepaald worden maar de twee uiterste punten. Dit zullen dan de pixels zijn die op de rand van het scherm liggen. Hiervoor wordt van elk cirkelsegment het eerste en laatste element opgeslagen. Dit is meteen ook het tweede pluspunt van het startpunt, het zorgt ervoor dat de eerste en laatste pixel van de lijsten altijd de buitenste punten van een boord of diagonaal zijn. Dit stelt het mogelijk de lijst waarover geïtereerd moet worden te reduceren tot 6 pixels (2 punten per segment, 2 boorden en 1 diagonaal). De twee pixels die het verst van elkaar gelegen zijn zoeken in deze lijst is veel efficïenter dan alle punten van de segmenten te moeten overlopen. Door de offset en andere opgestapelde afrondingen kan het mogelijk zijn dat  kan het zijn dat op de rechte vanuit elke van deze punten naar de hulphoek pixels gekruist worden die niet dezelfde id hebben als dat punt van waaruit de rechte werd opgesteld. Indien dat het geval is wordt de dichtsbijzijnde pixel genomen op dat cirkelsegment waarvoor dit wel mogelijk is. \paragraph{Tweede hulpunt} Afhankelijk van de onderlinge ligging tussen het te reconstrueren hoekpunt en de hulphoek moet beslist worden welke van de twee berekende punten verder nodig zal zijn voor de reconstructie. Dit kan als volgt achterhaald worden. Ter referentie wordt eerst de rechte tussen hulp- en overstaande hoek van de te reconstrueren hoek bepaald. Dan wordt voor de twee te reduceren punten de rechte opgesteld met de hulphoek. Degene die de grootste hoek vormt samen met de referentie rechte, bevat het te behouden punt en deze vormt dan ook de tweede constructielijn. De nieuwe hoek wordt dan bekomen door het snijpunt van de twee constructielijnen te bepalen \cite{intersectie}. Om de id van de nieuw bepaalde hoek te bepalen, kan gekeken worden naar de tegenovergestelde hoek die reeds gekend zal zijn. Is die roze dan is de nieuwe hoek geel en vice versa. 
		
		\begin{figure}[H]
			\center
			\includegraphics[width=0.8\textwidth]{img/screenReconstruction.jpg}
			\caption{Reconstructie van een hoekpunt. Zwarte punten zijn gebruikt, rode zijn ook berekend maar waren in deze situatie niet bruikbaar. Het rode vlak stelt de afdekking van de hoek voor.}
			\label{html}
			\label{scherm}
		\end{figure}
		
	

\section{Identificatie}\label{sec:identificatie}
 \subsection{Barcode}
Om de verschillende schermen te identificeren wordt gebruik gemaakt van een kleuren barcode. De barcode bestaat uit een herhalend patroon van 5 unieke kleuren telkens gevolgd door een witte lijn. Door het gebruik van deze witte lijn weet het algoritme waar het patroon eindigt en de volgende sequentie terug opnieuw begint. Het detecteren van deze 5 kleuren gebeurt aan de hand van opgestelde HSL ranges. Voor de identificatie van de slaves wordt er dus een unieke combinatie van deze 5 kleuren weergegeven. Deze vormt dan een unieke vijfcijferige code die gelinkt kan worden aan de bijhorende slave. Dit zorgt ervoor dat we in theorie een totaal van $5!$ ($=120$) verschillende schermen op één moment kunnen detecteren. Herhaling van het patroon heeft als resultaat dat bij overlap het scherm nog steeds geïdentificeerd kan worden. Het algoritme zal twee keer over alle pixels itereren. Hierdoor heeft het algorirmte een tijdscomplexiteit van
\[O(2mn)=O(mn)\]
met m en n de dimensies van het island waarin de barcode gelezen wordt. Het algoritme gaat een keer horizontaal en een keer verticaal om zo in alle mogelijke orientaties van het scherm de barcode te kunnen lezen. Vervolgens worden de HSL waarden van deze pixels bekeken om de overeenkomstige kleur van elke pixel te achterhalen. Wanneer een HSL waarde binnen de gewenste range valt, wordt het overeenkomstig cijfer opgeslaan in een lijst. Bij het bereiken van een witte lijn weet het algoritme dat het aan het einde van het patroon is. Wanneer dit het geval is wordt het inlezen van de volledige vijfcijferige code gecontroleerd op volledigheid. Zo niet wordt de lijst leeg gehaald en zoekt het algoritme verder. Het herhalend patroon is dus essentieel aan het correct inlezen van de barcode. Een groter aantal herhalingen stemt overeen met meer kansen op een mogelijke detectie, maar stemt ook overeen met een kleinere oppervlakte per herhaling. Deze kleinere oppervlakte is dan weer nadelig voor detectie omdat hiermee de kans stijgt dat een kleur niet gededecteerd wordt. Nadat het algoritme over alle pixels geweest is, wordt de ratio berekend tussen de code die het meeste keer gelezen is en de totale aantal codes die zijn gelezen. Deze ratio bepaalt dan of de code die door de horizontale iteratoe of die door de verticale iteratei het meest gelezen is, gebruikt wordt.

\subsection{Verdere verbeteringen}
Zoals al reeds vermeld bij het deel over kleurdetectie is het individueel dedecteren van zoveel verschillende kleuren geen goed idee. Dit is dan ook de reden dat andere opties reeds bekeken worden. Zoals ook reeds vermeld zou een eerste optie zijn om te kijken naar het contrast tussen de opeenvolgende kleuren in plaats van de kleuren apart te dedecteren. Een andere optie die bekeken wordt is om gewoon over te schakelen naar een zwart wit barcode waarbij eerder ook gebruik zal gemaakt worden van het contrast tussen zwart en wit dan de detectie ervan.


\input{Transformatie voor image-show}

\section{Triangulatie} \label{sec:triangulatie}
Wanneer de schermen herkent en geïdentificeerd zijn, worden ze aan elkaar gelinkt door middel van een triangulatie. Het project gebruikt een Delaunay triangulatie. Dit is een speciale vorm waarbij de kleinste hoek gemaximaliseert wordt en waarbij de driehoeken niet overlappen \cite{delaunaywiki}.
Er wordt gebruik gemaakt van het Bowyer-Watson algoritme. \cite{Bowyer-WatsonWiki} Het heeft een tijdscomplexiteit van $O(n^2)$, dit is zeker niet de beste complexiteit om een Delaunay triangulatie te berekenen. Aangezien in de toepassing maximaal 120 schermen gebruikt kunnen worden, voldoet $O(n^2)$. Met de eenvoudige implementatie is dit dan ook een voordehandliggende keuze.

\subsection{Bowyer-Watson}
Bowyer-Watson gaat er van uit dat punten enkel worden toegevoegd in een al bestaande Delaunay triangulatie. Als eerste worden er twee superdriehoeken gezocht. Deze driehoeken zullen alle te trianguleren punten bevatten. De implementaties waarop het algoritme is gebaseerd \cite{Bowyer-WatsonWiki} \cite{bowyer-watsonImplementation} stelden beiden een `superdriehoek' voor, zie figuur \ref{bowyer-watson-a} Echter is het eenvoudiger om een omkaderende vierhoek te vormen en deze op te delen in twee driehoeken. Vervolgens worden alle punten één voor één toegevoegd.

Voor elk punt worden alle driehoeken gezocht waarvan het punt in de omschreven cirkel zit. Wanneer twee driehoeken eenzelfde zijde delen, wordt deze verwijderd. Alle punten van de omschreven veelhoek van de twee driehoeken worden nu verbonden met het toegevoegde punt, zie figuur \ref{bowyer-watson-b}. Met deze werkwijze zal er op elk moment een Delaunay triangulatie zijn en moeten de driehoeken achteraf niet meer overlopen worden. Met het gevolg dat het data management voor deze methode minder complex is.

Eens alle punten toegevoegd zijn, worden de driehoeken die één of meer hoeken van de omkaderende vierhoek bevatten verwijderd, zie figuur \ref{bowyer-watson-c}. Aangezien deze driehoeken aan de buitenkant liggen is dit toegestaan. Er zal een Delaunay triangulatie overblijven van alle onderzochte punten.

\begin{figure}[H]
	\center
	\begin{subfigure}{0.4\textwidth}
		\includegraphics[width=\textwidth]{img/bowyer-watson_superdriehoek}
		\caption{De rode superdriehoek waarin alle punten zullen worden toegevoegd.}
		\label{bowyer-watson-a}
	\end{subfigure}
	\begin{subfigure}{0.4\textwidth}
		\includegraphics[width=\textwidth]{img/bowyer-watson_nieuwpunt}
		\caption{Een nieuw punt wordt toegevoegd. In geel de omschreven cirkels. De gestippelde zijde wordt verwijderd, de groene toegevoegd.}
		\label{bowyer-watson-b}
	\end{subfigure}
		\begin{subfigure}{0.4\textwidth}
		\includegraphics[width=\textwidth]{img/bowyer-watson_verwijderen}
		\caption{In rood alle driehoeken verbonden met de superdriehoek, deze worden uiteindelijk verwijderd.}
		\label{bowyer-watson-c}
	\end{subfigure}
	\caption{Het Bowyer-Watson algoritme \cite{Bowyer-WatsonWiki}}
	\label{bowyer-watson}
\end{figure}

\subsection{Valkuilen}
In theorie kan er van elke opstelling waarbij de punten niet allemaal colineair zijn een triangulatie worden gevonden, zie figuur \ref{colineair}. Echter door afronding bij de berekeningen zal er bij bijna colineaire punten geen juiste  configuratie gevonden worden, zie figuur \ref{almost_colineair}. 

\begin{figure}[H]
	\center
	\includegraphics[width=0.4\textwidth]{img/colineair}
	\caption{Van colineaire punten kan geen triangulatie gevonden worden.}
	\label{colineair}
\end{figure}
\begin{figure}[H]
	\center
	\includegraphics[width=0.4\textwidth]{img/almost_colinair}
	\caption{Van bijna colineaire punten kan geen triangulatie gevonden worden door afrondingsfouten bij berekeningen.}
	\label{almost_colineair}
\end{figure}



\section{Animation}


\subsection{Sprites}

De animatie van de kat of muis wordt bekomen door het snel achter elkaar tekenen van delen van een sprite sheet. Deze sheet bevat in dit geval 7 frames, zie figuur \ref{fig:cat} en \ref{fig:mouse}. De breedte van de sheet zal door het aantal frames worden gedeeld om zo met een drawmethode elke deel apart en achter elkaar te displayen. Op deze manier creërt men de illusie van beweging op een zeer simpele manier. In dit geval loopt de kat of muis naar rechts maar met een spiegel methode kan de andere richting bekomen worden. De sprite zal ook worden geroteerd om in de correcte richting te bewegen. Meerdere muizen zullen achter elkaar lopen zodat op verschillende schermen sprites worden afgebeeld, zie figuur \ref{fig:schermen}. Dit werkt aan de hand van een stack waarin zich de vorige posities van de eerste kat bevinden. Na de laatste drawmethode zal het eerste element van de lijst worden verwijderd. Alle dieren zullen dus hetzelfde pad nemen.

\begin{figure}[H]
\centering
\includegraphics[scale=0.2]{img/cat2.png}
\caption{Kat sprite sheet \cite{catsprite}}
\label{fig:cat}
\end{figure}

\begin{figure}[H]
\centering
\includegraphics[scale=0.8]{img/mouse2_trans.png}
\caption{Muis sprite sheet \cite{mousesprite}}
\label{fig:mouse}
\end{figure}


\newpage
\subsection{Delaunay}

Voor de animatie wordt gebruik gemaakt van de reeds geschreven Delaunay-triangulatie. Deze zal het pad vormen waarover de kat of muis zal lopen. Wanneer het dier zich dicht genoeg bij het endPoint bevindt zal deze de firstPoint worden en zullen de buren ervan opgevraagd worden. Er zal willeukeurig een punt worden gekozen als nieuw endPoint. Dit punt kan niet het firstPoint van de vorige beweging zijn. Dit wil zeggen dat de animatie nooit terug op zijn stappen komt. Enkel als de triangulatie een rechte lijn vormt zal de sprite op en neer lopen. 

\subsection{Server/Client side}

De server zal simpelweg de basisinfo doorgeven aan de client, zelf de berekeningen maken voor de volgende posities en de triangulatie bijhouden. De client krijgt enkel een dictionary met de xy-positie, hoek, frame en een boolean voor spiegeling en heeft geen weet van de triangulatie.

\begin{figure}[H]
\centering
\includegraphics[scale=0.5]{img/schermen.png}
\caption{Theoretische animatie met verschillende schermen}
\label{fig:schermen}
\end{figure}



\section{Besluit}
Er is een applicatie ontwikkeld waarin een master verschillende cliëntschermen kan detecteren en manipuleren. Het framework houdt rekening met uitbreiding naar verschillende {\it kamers}. Voor video wordt er gestreamd in plaats van volledig gedownload. Met behulp van synchronisatie zal een video of een aftelklok gelijk lopen op elk scherm. Enkel bij uitzonderlijke gevallen waarbij de {\it ping} toevallig fout wordt gemeten (door bv. actieve achtergrondprocessen) kan er een synchronisatiefout optreden.\\[3mm]
Het master device kan vanuit één enkele foto alle schermen herkennen. Hierbij wordt gebruik gemaakt van een bepaald herkenningspatroon. Bij sterke reflectie, grote 3d-draaiing en verschillende kleurweergave van de schermen kan het mislopen. Schermen worden dan niet herkend en zullen dus niet kunnen samenwerken met de herkende schermen. Een oplossing wordt hiervoor gezocht in de vorm van kleurverschil i.p.v. kleurdetectie. Hiervoor zijn er al plots opgesteld om de reikwijdte van de verschillende kleuren te determineren en zo de kleurschema's beter en efficiënter te gebruiken.\\[3mm]
Het reconstrueren van schermen, wanneer er een deel van het scherm wordt bedekt, vertrouwt erop dat er 2 aanliggende hoeken en het middelpunt worden gevonden. Met behulp van een transformatiematrix wordt de ligging en draaiing van het scherm bepaald. Hierdoor kan een afbeelding met juiste transformatie geprojecteerd worden op elk scherm. Deze heeft echter vier hoeken nodig en bouwt verder op de reconstructie van het scherm.\\[3mm]
Als laatste is er nog de animatie die zeer nauw samenhangt met de triangulatie. Hiervoor is een Delaunay-triangulatie gekozen omdat deze uniek is, niet overlappend en een maximale kleinste hoek nastreeft. Wanneer alle punten (bijna) collineair zijn, zal er door afronding geen triangulatie gevonden worden, de animatie kan dan ook niet worden afgespeeld.\\[3mm]
De applicatie heeft een brede waaier aan functionaliteiten. Doordat er maar één foto nodig is om de schermen te herkennen, is het plaatsen van de schermen nauwkeurig maar kan de identificatie sneller mislopen. Alle opdrachten, exclusief de cat caster, zijn geïmplementeerd. Met enkel nog een paar kleine {\it bugs}, waarnaar gestreefd wordt deze op te lossen, is er sprake van een geslaagde opdracht.

\newpage
\addcontentsline{toc}{section}{Referenties}
\bibliographystyle{unsrt}
\bibliography{Bibliography}

\newpage
\appendix
\section{Overzicht algoritme}
\subsection{Algemeen}
\begin{figure}[H]
	\centering
	\includegraphics[scale= 0.5]{img/algoOverview.png}
	\caption{}
	\label{bijlageA1}
\end{figure}
\newpage
\subsection{Hoekdetectie en filtering}
\begin{figure}[H]
	\centering
	\includegraphics[scale= 1]{img/DetectAndFilterCorners.png}
	\caption{}
	\label{bijlageA2}
\end{figure}
\newpage
\subsection{Reconstructie}
\begin{figure}[H]
	\centering
	\includegraphics[scale= 0.55]{img/ReconstructionFlowchart.png}
	\caption{}
	\label{bijlageA3}
\end{figure}
\newpage

\section{Kleuren plots}

\subsection{RGB plots}

\begin{figure}[h!]
	\centering
	\begin{minipage}{0.5\textwidth}
		\centering
		\includegraphics[width=0.9\textwidth]{img/rgbRed.png}
		\captionsetup{width=0.9\textwidth}
		\captionof{figure}{RGB plot voor de kleur rood.}
		\label{rgbRedPlot}
	\end{minipage}%
	\begin{minipage}{0.5\textwidth}
		\centering
		\includegraphics[width=0.9\textwidth]{img/rgbYellow.png}
		\captionsetup{width=0.9\textwidth}
		\captionof{figure}{RGB plot voor de kleur geel.}
		\label{rgbYellowPlot}
	\end{minipage}
\end{figure}

\vspace{5mm}

\begin{figure}[h!]
	\centering
	\begin{minipage}{0.5\textwidth}
		\centering
		\includegraphics[width=0.9\textwidth]{img/rgbGreen.png}
		\captionsetup{width=0.9\textwidth}
		\captionof{figure}{RGB plot voor de kleur groen.}
		\label{rgbGreenPlot}
	\end{minipage}%
	\begin{minipage}{0.5\textwidth}
		\centering
		\includegraphics[width=0.9\textwidth]{img/rgbBlueGreen.png}
		\captionsetup{width=0.9\textwidth}
		\captionof{figure}{RGB plot voor de kleur cyaan.}
		\label{rgbBlueGreenPlot}
	\end{minipage}
\end{figure}

\vspace{5mm}

\begin{figure}[h!]
	\centering
	\begin{minipage}{0.5\textwidth}
		\centering
		\includegraphics[width=0.9\textwidth]{img/rgbBlue.png}
		\captionsetup{width=0.9\textwidth}
		\captionof{figure}{RGB plot voor de kleur blauw.}
		\label{rgbBluePlot}
	\end{minipage}%	
	\begin{minipage}{0.5\textwidth}
		\centering
		\includegraphics[width=0.9\textwidth]{img/rgbPink.png}
		\captionsetup{width=0.9\textwidth}
		\captionof{figure}{RGB plot voor de kleur magenta.}
		\label{rgbPinkPlot}
	\end{minipage}
\end{figure}

\subsection{Histogrammen}

\begin{figure}[h!]
	\centering
	\begin{minipage}{0.5\textwidth}
		\centering
		\includegraphics[width=0.9\textwidth]{img/hueHistRed.png}
		\captionsetup{width=0.9\textwidth}
		\captionof{figure}{Histogram van de tint in functie van het aantal waargenomen pixels voor de kleur rood.}
		\label{histRed}
	\end{minipage}%
	\begin{minipage}{0.5\textwidth}
		\centering
		\includegraphics[width=0.9\textwidth]{img/hueHistYellow.png}
		\captionsetup{width=0.9\textwidth}
		\captionof{figure}{Histogram van de tint in functie van het aantal waargenomen pixels voor de kleur geel.}
	\label{histYellow}
	\end{minipage}
\end{figure}

\vspace{1mm}

\begin{figure}[h!]
	\centering
	\begin{minipage}{0.5\textwidth}
		\centering
		\includegraphics[width=0.9\textwidth]{img/hueHistGreen.png}
		\captionsetup{width=0.9\textwidth}
		\captionof{figure}{Histogram van de tint in functie van het aantal waargenomen pixels voor de kleur groen.}
		\label{histGreen}
	\end{minipage}%
	\begin{minipage}{0.5\textwidth}
		\centering
		\includegraphics[width=0.9\textwidth]{img/hueHistBlueGreen.png}
		\captionsetup{width=0.9\textwidth}
		\captionof{figure}{Histogram van de tint in functie van het aantal waargenomen pixels voor de kleur cyaan.}
	\label{histBluegreen}
	\end{minipage}
\end{figure}

\vspace{1mm}

\begin{figure}[h!]
	\centering
	\begin{minipage}{0.5\textwidth}
		\centering
		\includegraphics[width=0.9\textwidth]{img/hueHistBlue.png}
		\captionsetup{width=0.9\textwidth}
		\captionof{figure}{Histogram van de tint in functie van het aantal waargenomen pixels voor de kleur blauw.}
		\label{histBlue}
	\end{minipage}%
	\begin{minipage}{0.5\textwidth}
		\centering
		\includegraphics[width=0.9\textwidth]{img/hueHistPink.png}
		\captionsetup{width=0.9\textwidth}
		\captionof{figure}{Histogram van de tint in functie van het aantal waargenomen pixels voor de kleur magenta.}
	\label{histPink}
	\end{minipage}
\end{figure}

\subsection{HSL plots}

\subsubsection{3D}

\begin{figure}[h!]
	\centering
	\begin{minipage}{0.5\textwidth}
		\centering
		\includegraphics[width=0.9\textwidth]{img/hsl3DRed.png}
		\captionsetup{width=0.9\textwidth}
		\captionof{figure}{HSL plot voor de kleur rood.}
		\label{hsl3DRedPlot}
	\end{minipage}%
	\begin{minipage}{0.5\textwidth}
		\centering
		\includegraphics[width=0.9\textwidth]{img/hsl3DYellow.png}
		\captionsetup{width=0.9\textwidth}
		\captionof{figure}{HSL plot voor de kleur geel.}
		\label{hsl3DYellowPlot}
	\end{minipage}
\end{figure}

\vspace{5mm}

\begin{figure}[h!]
	\centering
	\begin{minipage}{0.5\textwidth}
		\centering
		\includegraphics[width=0.9\textwidth]{img/hsl3DGreen.png}
		\captionsetup{width=0.9\textwidth}
		\captionof{figure}{HSL plot voor de kleur groen.}
		\label{hsl3DGreenPlot}
	\end{minipage}%
	\begin{minipage}{0.5\textwidth}
		\centering
		\includegraphics[width=0.9\textwidth]{img/hsl3DBlueGreen.png}
		\captionsetup{width=0.9\textwidth}
		\captionof{figure}{HSL plot voor de kleur cyaan.}
		\label{hsl3DBlueGreenPlot}
	\end{minipage}
\end{figure}

\vspace{5mm}

\begin{figure}[h!]
	\centering
	\begin{minipage}{0.5\textwidth}
		\centering
		\includegraphics[width=0.9\textwidth]{img/hsl3DBlue.png}
		\captionsetup{width=0.9\textwidth}
		\captionof{figure}{HSL plot voor de kleur blauw.}
		\label{hsl3DBluePlot}
	\end{minipage}%	
	\begin{minipage}{0.5\textwidth}
		\centering
		\includegraphics[width=0.9\textwidth]{img/hsl3DPink.png}
		\captionsetup{width=0.9\textwidth}
		\captionof{figure}{HSL plot voor de kleur magenta.}
		\label{hsl3DPinkPlot}
	\end{minipage}
\end{figure}

\subsubsection{2D}

\begin{figure}[h!]
	\center
	\includegraphics[width=\textwidth]{img/hslRed.png}
	\caption{Scatter plots in functie van tint en saturatie, alsook in functie van tint en lichtheid voor de kleur rood.}
	\label{hslRedPlot}
\end{figure}

\vspace{25mm}

\begin{figure}[h!]
	\center
	\includegraphics[width=\textwidth]{img/hslYellow.png}
	\caption{Scatter plots in functie van tint en saturatie, alsook in functie van tint en lichtheid voor de kleur geel.}
	\label{hslYellowPlot}
\end{figure}

\begin{figure}[h!]
	\center
	\includegraphics[width=\textwidth]{img/hslGreen.png}
	\caption{Scatter plots in functie van tint en saturatie, alsook in functie van tint en lichtheid voor de kleur groen.}
	\label{hslGreenPlot}
\end{figure}

\begin{figure}[h!]
	\center
	\includegraphics[width=\textwidth]{img/hslBlueGreen.png}
	\caption{Scatter plots in functie van tint en saturatie, alsook in functie van tint en lichtheid voor de kleur cyaan.}
	\label{hslBlueGreenPlot}
\end{figure}

\begin{figure}[h!]
	\center
	\includegraphics[width=\textwidth]{img/hslBlue.png}
	\caption{Scatter plots in functie van tint en saturatie, alsook in functie van tint en lichtheid voor de kleur blauw.}
	\label{hslBluePlot}
\end{figure}

\begin{figure}[h!]
	\center
	\includegraphics[width=\textwidth]{img/hslPink.png}
	\caption{Scatter plots in functie van tint en saturatie, alsook in functie van tint en lichtheid voor de kleur magenta.}
	\label{hslPinkPlot}
\end{figure}

\newpage

\end{document}