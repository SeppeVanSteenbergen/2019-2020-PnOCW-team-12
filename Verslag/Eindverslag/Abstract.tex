Het laten samenwerken van verschillende schermen van computers, tablets, smartphones en alle andere soorten, met toegang tot een moderne web-browser, om samen een beeld volledig beeld te vormen en dit alles geconfigureerd door één master device is het doel van deze bachelorproef. De devices zijn samen verbonden aan een web-applicatie en worden geconfigureerd door door een foto te nemen met het master device door de gebruiker. Met of zonder overlap, sommige dicht en sommige schermen verder van de camera, recht of gekanteld, het eindresultaat dat de schermen weergeven moet vlak tonen vanuit de configuratiepositie van de gebruiker.

Het start allemaal met het detecteren van de verschillende schermen. Voor ons was het vanaf het eerste moment het doel om het volledige resultaat te bereiken door het maken van één enkele foto door een standaard smartphone camera met alle gewenste schermen aanwezig. Dit zou het lastige probleem van een inconsistente camera positie van verschillende foto’s meteen elimineren. Al vanaf de eerste resultaten werkte we met voorbepaalde kleuren en filteren op geassocieerde color ranges en vormen die op de slaves worden afgebeeld voor de detectie en maskers van de te detecteren schermen te vormen. Met behulp van onderzoek naar HSL colorranges en floodfill worden de maskers en individuele associatie van de schermen afgeleid uit de foto. Verdere algoritmes voor hoekdetectie op elke scherm en afgedekte delen van de schermen te reconstrueren gebeuren op deze maskers. De link tussen de gevonden schermen en clients die verbonden zijn met de server via de web-applicatie worden gelinkt door het lezen van een unieke kleuren-barcode die overeen komt met een client-ID. Met deze overeenkomende informatie tussen client en gevonden scherm wordt er een perspectief matrix afgeleid om een correct resultaat op elk scherm af te beelden dat zowel de continuïteit en visuele vlakheid van het resultaat behoud en garandeerd.

Tijdens de eindfases en het verder drijven van de omstandigheden is het duidelijk geworden dat de kleurendetectie de grootste struikelblok was van de volledige methode. Deze aanpak viel al snel uit elkaar bij relatief kleine uitwijkingen in de omgevingsbelichting en reflecties in de schermen. Enkel bij schermen en camera’s die gebruikt werden voor het afstellen van de kleurdetectie kregen we een consistent resultaat. Bij het terugblikken is een makkelijke oplossing om de barcode voor indentificatie te beperken tot grijswaarden om zo een grotere range voor de kleurendetectie te laten. Zo worden er minder kleuren binnen het HSL spectrum door elkaar gehaald en is er meer ruimte binnen het spectrum voor elke kleur. De ontworpen algoritmes voor hoekdetectie en reconstructie werken bij een correct gevormd masker naar een verlangend resultaat.
