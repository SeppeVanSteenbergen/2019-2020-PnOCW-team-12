Het vorige verslag behandelde de kleurdetectie, met focus op de te herkennen kleur en de invloeden van omgevingsfactoren. Bij het herkennen van kleuren spelen oneffenheden in de foto's ook een rol, het verslag doet hiernaar verder onderzoek. Een foto is onderhevig aan verschillende factoren; Zo zal een artificiële intelligentie, die nu vaak standaard is ingebouwd in smartphones, de foto al bewerken. Lichtinval, een ander scherm of zelfs een vuil scherm, maakt dat de kleuren niet egaal worden weergegeven.

In sectie \ref{sec:achtergrond} komen verscheidene filters aan bod die de oneffenheden en ruwheden in een foto kunnen reduceren. Het zijn deze filters waar het onderzoek zich op focust. Op welke manier doen ze dit, hoe zijn deze geïmplementeerd... Vervolgens diept het verslag de toegepaste methode uit, waarna het meer uitleg geeft over de bevindingen van dit onderzoek. Een eerste zeer belangrijke factor is de tijdsconsumptie van de filter. Weegt de toeneming van uitvoeringstijd op tegen de verbetering van het resultaat? Daarnaast kan gekeken worden naar foto's die initieel niet correct werden gedetecteerd. Zorgt het toepassen van een filter ervoor dat deze wel correct wordt gedetecteerd? Welke filter geeft dan het beste resultaat? Zorgt het aanpassen van de kerngrootte voor betere detectie, en heeft dit effect op de uitvoeringstijd? Ten slotte wordt gekeken of er bevindingen toepasbaar zijn binnen het project.